\documentclass{article}

%%%%%%%%%%%%%%%%%%%%%%%%%%%%%%%%%%%%%%%%%%%%%%%%%%%%%%%%%%%%%%%%%%%%%%%%%%%%%%%%
%% package setup
%%%%%%%%%%%%%%%%%%%%%%%%%%%%%%%%%%%%%%%%%%%%%%%%%%%%%%%%%%%%%%%%%%%%%%%%%%%%%%%%


\usepackage[shortlabels]{enumitem}
\usepackage{amsfonts,amsmath, soul, matlab-prettifier, bm, amsthm,enumitem,amssymb,multirow,float,mathtools,bbm,array,varwidth,hyperref,bm}
\usepackage[all]{xy}
\usepackage[margin=0.9in]{geometry}
\usepackage{graphicx}
\usepackage{mathtools, caption, dsfont, tikz-cd}
\usepackage{wrapfig}

% tikzpicture

\usepackage{pgfplots}
\pgfplotsset{compat=1.15}
\usepackage{mathrsfs}
\usetikzlibrary{arrows}


\mathchardef\mhyphen="2D


\raggedbottom

%%%%%%%%%%%%%%%%%%%%%%%%%%%%%%%%%%%%%%%%%%%%%%%%%%%%%%%%%%%%%%%%%%%%%%%%%%%%%%%%
%% operators and symbols
%%%%%%%%%%%%%%%%%%%%%%%%%%%%%%%%%%%%%%%%%%%%%%%%%%%%%%%%%%%%%%%%%%%%%%%%%%%%%%%%

% operators
\newcommand{\Hom}{\mathrm{Hom}}
\newcommand{\Ext}{\mathrm{Ext}}
\newcommand{\Ker}{\mathrm{Ker}}
\newcommand{\Pic}{\mathrm{Pic}}
\newcommand{\comp}{\mathrm{comp}}
\newcommand{\Proj}{\mathrm{Proj}}
\newcommand{\rk}{\mathrm{rk}}
\newcommand{\Spec}{\mathrm{Spec}}
\newcommand{\Sym}{\mathrm{Sym}}
\newcommand{\Frob}{\mathrm{Frob}}
\newcommand{\Gal}{\mathrm{Gal}}
\newcommand{\GL}{\mathrm{GL}}
\newcommand{\SL}{\mathrm{SL}}
\newcommand{\Ind}{\mathrm{Ind}}
\newcommand{\Rep}{\mathrm{Rep}}
\newcommand{\Aut}{\mathrm{Aut}}
\newcommand{\Res}{\mathrm{Res}}
\newcommand{\Smo}{\mathrm{Smo}}
\newcommand{\Span}{\mathrm{Span}}
\newcommand{\Frac}{\mathrm{Frac}}
\newcommand{\supp}{\mathrm{supp}}
\newcommand{\End}{\mathrm{End}}
\newcommand{\St}{\mathrm{St}}
\newcommand{\Top}{\mathrm{Top}}
\newcommand{\Ab}{\mathrm{Ab}}
\newcommand{\Set}{\mathrm{Set}}
\newcommand{\ad}{\mathrm{ad}}
\newcommand{\hht}{\mathrm{ht}}

\newcommand{\PGL}{\mathrm{PGL}}
\newcommand{\PSL}{\mathrm{PSL}}
\newcommand{\Psh}{\mathrm{Psh}}
\newcommand{\Sh}{\mathrm{Sh}}
\newcommand{\Id}{\mathrm{Id}}

% greek 
\newcommand{\calpha}{\check{\alpha}}
\newcommand{\cPhi}{\check{\Phi}}
\newcommand{\cbeta}{\check{\beta}}

% shortcuts
\newcommand{\slii}{\mathrm{sl}_2}
\newcommand{\sliii}{\mathrm{sl}_3}
\newcommand{\eW}{\widetilde{W_a}}


% mathcal
\newcommand{\cO}{\mathcal{O}}

% mathbb
\newcommand{\CC}{\mathbb{C}}
\newcommand{\FF}{\mathbb{F}}
\newcommand{\NN}{\mathbb{N}}
\newcommand{\PP}{\mathbb{P}}
\newcommand{\QQ}{\mathbb{Q}}
\newcommand{\RR}{\mathbb{R}}
\newcommand{\ZZ}{\mathbb{Z}}
\newcommand{\GG}{\mathbb{G}}
\newcommand{\adele}{\mathbb{A}}
\newcommand{\pp}{\mathfrak{p}}
\newcommand{\nn}{\mathfrak{n}}
\newcommand{\mm}{\mathfrak{m}}

% mathfrak
\newcommand{\fg}{\mathfrak{g}}
\newcommand{\fm}{\mathfrak{m}}

% shortcuts
\newcommand{\CG}{C_c^{\infty}(G)}
\newcommand{\cInd}{c\mhyphen\mathrm{Ind}}

\newcommand{\norm}[1]{\left\lVert#1\right\rVert}
\newcommand{\hatv}[1]{\overset{\vee}{\mathstrut#1}}

\DeclareMathOperator{\Ima}{Im}

\linespread{1.5}

\theoremstyle{plain}
\newtheorem{theorem}{Theorem}[section]
\newtheorem{question}[theorem]{Question}
\newtheorem{proposition}[theorem]{Proposition}
\newtheorem{convention}[theorem]{Convention}
\newtheorem{lemma}[theorem]{Lemma}
\newtheorem{cor}[theorem]{Corollary}
\newtheorem{algo}[theorem]{Algorithm}
\theoremstyle{definition}
\newtheorem{definition}[theorem]{Definition}
\newtheorem{notn}[theorem]{Notation}
\newtheorem{remark}[theorem]{Remark}
\newtheorem{example}[theorem]{Example}
\newtheorem{examples}[theorem]{Examples}
\newtheorem{fact}[theorem]{Fact}
\newtheorem*{hypothesis}{Hypothesis}
\newtheorem*{exercise}{Exercise}


\title{Bruhat-Tits Theory}
\author{Albert Lopez Bruch}

\begin{document}
	\maketitle
	\pagenumbering{arabic}
	
    \section{The Weyl Group}\label{chp:weylgroup}
    \subsection{Root Systems}
Let $E$ be a finite dimensional vector space over $\RR$ with an inner product $(\cdot,\cdot):E\times E\to\RR$. For each $\lambda\in E$, we let $H_\lambda=\{\mu\in E:(\mu,\lambda)=0\}$ be the hyperplane perpendicular to $\lambda$ and let $$w_\lambda(\mu)=\mu-\frac{2(\mu,\lambda)}{(\lambda,\lambda)}\lambda,\quad \mu\in E$$ be the reflection along $H_\lambda$.
\begin{definition}
    A subset $\Phi\subset E$ is a \textit{root system} if
    \begin{enumerate}
        \item $\Phi$ is finite and $0\not\in\Phi$.
        \item $\Phi$ spans $E$.
        \item For all $\alpha,\beta\in\Phi$, $\frac{2(\alpha,\beta)}{(\alpha,\alpha)}\in\ZZ$.
        \item For all $\alpha\in\Phi$, $w_\alpha(\Phi)=\Phi$.
        \item If $\alpha\in\Phi$, then $\Phi\cap\ZZ\alpha=\{\alpha,-\alpha\}$.
    \end{enumerate}
\end{definition} 

It is clear from the definition that the reflections $w_\alpha$ induce automorphisms of the root system. 

\begin{definition}
    The \textit{Weyl group} $W_\Phi$ of the root system $\Phi$ is defined as the group generated by $\{w_\alpha:\alpha\in\Phi\}$.
\end{definition}

Since $E$ comes equipped with an inner product, there is a canonical isomorphism $E\cong E^*$ given by $\lambda\mapsto\lambda^*$ where $\lambda^*(\mu)=(\lambda,\mu)$. Furthermore, the inner product naturally extends to $E^*$ by declaring that the isomorphism above is an isometry; that is, by defining $(\lambda^*,\mu^*)=(\lambda,\mu)$. 

Furthermore, the isomorphism $E\rightarrow E^{**}$ given by $\lambda\mapsto\lambda^{**}$ coincides with the evaluation map $\mathrm{ev}:E\rightarrow E^{**}$ satisfying $\mathrm{ev}(\lambda)(f)=f(\lambda)$ for all $\lambda\in E$ and $f\in E^*$. Indeed, if $\mu\in E$ satisfies that $\mu^*=f$, then 
$$\lambda^{**}(f)=(\lambda^*,f)=(\lambda^*,\mu^*)=(\lambda,\mu)=\mu^*(\lambda)=f(\lambda)=\mathrm{ev}(\lambda)(f).$$

Given $\alpha\in\Phi$, we can define $\check{\alpha}=\frac{2\alpha^*}{(\alpha,\alpha)}\in E^*$. Then conditions $3.$ and $4.$ above can be rephrased as
\begin{enumerate}[start=3]
    \item For all $\alpha,\beta\in\Phi$, $\calpha(\beta)\in\ZZ$.
    \item For all $\alpha,\beta\in\Phi$, $w_\alpha(\beta)=\beta-\calpha(\beta)\beta\in\Phi$.
\end{enumerate}

This notation is useful due to the following fact:

\begin{lemma}
    Let $\cPhi=\{\calpha:\alpha\in\Phi\}\subset E^*$. Then $\cPhi$ is a root system of $E^*$, denoted the \textit{dual root system}.
\end{lemma}
\begin{proof}
    Clearly, $\cPhi$ is a finite subset, and if it does not span $E^*$, then there is some $\mu\in E\setminus\{0\}$ such that $\calpha(\mu)=0$ for all $\alpha\in\Phi$. This means that $(\mu,\alpha)=0$ for all $\alpha\in\Phi$, so $\mu$ lies in the orthogonal complement of $\mathrm{Span}_\RR(\Phi)=E$, a contradiction.

    Next, we note that if $\alpha,\beta\in\Phi$, then 
    $$(\calpha,\cbeta)=\frac{4(\alpha,\beta)}{(\alpha,\alpha)(\beta,\beta)},$$
    and therefore 
    $$\check{\calpha}(\cbeta)=\frac{2(\calpha,\cbeta)}{(\calpha,\calpha)}=\frac{2(\alpha,\beta)}{(\beta,\beta)}=\cbeta(\alpha)\in \ZZ.$$
    This calculation also shows that 
    $$w_{\calpha}(\cbeta)=\cbeta-\check{\calpha}(\cbeta)\calpha=\frac{2\beta^*}{(\beta,\beta)}-\frac{2(\alpha,\beta)}{(\beta,\beta)}\frac{2\alpha^*}{(\alpha,\alpha)}=\frac{2(\beta^*-\calpha(\beta)\alpha^*)}{(w_\alpha(\beta),w_\alpha(\beta))}=\frac{2w_\alpha(\beta)^*}{(w_\alpha(\beta),w_\alpha(\beta))}=\check{w_\alpha(\beta)}\in\cPhi.$$
    Finally, if $\calpha\in\cPhi$ and $c\calpha=\check{(\alpha/c)}\in\cPhi$ for some $c\in\RR$, then $\alpha/c\in\Phi$ so $c=\pm1$.
\end{proof}

\begin{proposition}
    The function
    \begin{align*}
        \phi:W_{\Phi}&\longrightarrow W_{\cPhi}\\
        w_\alpha&\longmapsto w_{\calpha}
    \end{align*}
    extended multiplicatively induces a well-defined isomorphism of groups.
\end{proposition}
\begin{proof}
    We claim that for any $w\in W_\Phi$, $\phi(w)=\check{w}$, where $\check{w}(\cbeta)=\check{w(\beta)}$ for any $\beta\in\Phi$. This is indeed the case for $w=w_\alpha$ for any $\alpha$ since $w_{\calpha}(\cbeta)=\check{w_\alpha(\beta)}$ for any $\beta\in\Phi$. 
    
    Suppose now that $\phi(w)=\check{w}$ for some $w\in W_\Phi$ and let $\alpha\in\Phi$. Then $$\phi(ww_\alpha)(\cbeta)=\phi(w)w_{\calpha}(\cbeta)=\check{w}(\check{w_\alpha(\beta)})=\check{ww_\alpha(\beta)}.$$
    Since $W_\Phi$ is generated by $\{w_\alpha:\alpha\in\Phi\}$, a simple inductive argument proves the claim. This description of $\phi(w)=\check{w}$ is generator-free, and hence $\phi$ is a well-defined group homomorphism. It is surjective since $W_{\cPhi}$ is generated by $\{w_{\calpha}:\calpha\in\cPhi\}$ and is injective since $\check{w}(\cbeta)=\cbeta$ for all $\beta\in\Phi$ if and only if $w(\beta)=\beta$ for all $\beta\in\Phi$.
\end{proof}

From now on, we will simply write $W$ whenever the underlying root system is clear from context.

\subsection{Properties of the Weyl Group}

Fix some root system $\Phi\subset E$, and for each $\alpha\in\Phi$, let $H_\alpha=\{\lambda\in E:(\alpha,\lambda)=0\}$ be the hyperplane perpendicular to $\alpha$. An element $\gamma\in E\setminus\cup_{\alpha\in\Phi}H_\alpha$ is said to be \textit{regular} and connected components of $E\setminus\cup_{\alpha\in\Phi}H_\alpha$ are called \textit{Weyl chambers}. 

Finally, we say that a subset $\Delta=\{\alpha_1,\ldots,\alpha_l\}\subset\Phi$ is an \textit{integral basis} if $\Delta$ is a basis of $E$ and any $\alpha=\sum_{i=1}^l c_i\alpha_i\in\Phi$ satisfies that either all $c_i\in\ZZ^{\geq 0}$ or all $c_i\in\ZZ^{\leq 0}$. If such a subset $\Delta$ exists, then we distinguish between \textit{positive roots} $\Phi^+$ (if all $c_i\in\ZZ^{\geq0}$) and \textit{negative roots} $\Phi^-$ (if all $c_i\in\ZZ^{\leq0}$).

The aim of this subsection is to prove the following theorem that describes the structure of the Weyl group.

\begin{theorem}\label{thm:weyl}
    Let $\Phi\subset E$ be a root system and let $W$ be its Weyl group. Then
    \begin{enumerate}[label=(\arabic*)]
        \item $\Phi$ contains an integral basis $\Delta=\{\alpha_1,\ldots,\alpha_l\}$.
        \item There is a 1-to-1 correspondence between integral basis and Weyl chambers.
        \item $W$ is generated by $\{w_{\alpha_1},\ldots,w_{\alpha_l}\}$.
        \item $W$ acts simply transitively on Weyl chambers (and also on integral basis).
    \end{enumerate}
\end{theorem}

\begin{proof}[Proof of Theorem \ref{thm:weyl}(1), (2) and (3)]
    Let $\gamma$ be some regular element and let $\Phi^+(\gamma)=\{\alpha\in\Phi:(\gamma,\alpha)>0\}$. Define $$\Delta(\gamma):=\{\alpha\in\Phi^+(\gamma):\text{ there are no }\beta_1,\beta_2\in\Phi^+(\gamma) \text{ such that }\alpha=\beta_1+\beta_2\}.$$ 
    Now suppose that not all $\alpha\in\Phi^+(\gamma)$ is a $\ZZ^{\geq0}$ linear combination of $\Delta(\gamma)$ and choose such a $\beta$ that minimizes $(\beta,\gamma)$. Since $\beta\not\in\Delta(\gamma)$, there are $\beta_1,\beta_2\in\Phi^+(\gamma)$ such that $\beta=\beta_1+\beta_2$. But $(\beta_i,\gamma)<(\beta,\gamma)$ for $i=1,2$ so $\beta_1,\beta_2$ are a $\ZZ^{\geq0}$ linear combination of $\Delta(\gamma)$, contradicting the choice of $\beta$. Hence, all $\alpha\in\Phi^+(\gamma)$ are a $\ZZ^{\geq0}$ linear combination of $\Delta(\gamma)$. Since $\Phi=\Phi^+(\gamma)\cup-\Phi^+(\gamma)$, all $\alpha=\sum_{i=1}^l c_i\alpha_i\in\Phi$ satisfies that either all $c_i\in\ZZ^{\geq 0}$ or all $c_i\in\ZZ^{\leq 0}$.

    Next, we need to show that $\Delta(\gamma)$ is a basis. Let $\alpha,\beta\in\Delta(\gamma)$ and suppose $(\alpha,\beta)>0$. Then $\pm(\alpha-\beta)\in\Phi$ so either $\alpha-\beta\in\Phi^+(\gamma)$ or $\beta-\alpha\in\Phi^+(\gamma)$. In the former case, $\alpha=(\alpha-\beta)+\beta$ and in the latter, $\beta=(\beta-\alpha)+\alpha$. Either case contradicts the definition of $\Delta(\gamma)$, so $(\alpha,\beta)\leq0$.
    Suppose that $$v=\sum_{i=1}^rc_i\alpha_i=\sum_{j=1}^sd_j\beta_j\in E$$ where $\Delta(\gamma)=\{\alpha_1,\ldots,\alpha_r,\beta_1,\ldots,\beta_s\}$ and all $c_i,d_j\geq0$. Then
    $$0\leq(v,v)=\sum_{i=1}^r\sum_{j=1}^sc_id_j(\alpha_i,\beta_j)\leq0,$$
    so $v=0$. Finally, by definition of $\Phi^+(\gamma)$,
    $$0=(\gamma,v)=\sum_{i=1}^{r}c_i(\gamma,\alpha_i)=\sum_{j=1}^{s}d_j(\gamma,\beta_j)$$
    implies that $c_i=d_j=0$ for all $1\leq i\leq r$ and $1\leq j\leq s$. This concludes the proof of (1).

    It is clear the definition of $\Phi^+(\gamma)$ only depends on the Weyl chamber $\gamma$ lies in, so the above proof gives a way to construct an integral basis from a Weyl chamber. Conversely, given an integral basis $\Delta=\{\alpha_1,\ldots,\alpha_l\}$ the intersection of half spaces $\cap_{i=1}^l\{\lambda\in E:(\lambda,\alpha_i)>0\}$ is a Weyl chamber (in particular, it is non-empty), denoted as $C(\Delta)$. It is clear that $\gamma\in C(\Delta(\gamma))$ and that if $\gamma\in C(\Delta)$, then $\Delta=\Delta(\gamma)$, so the constructions are inverses of each other. This concludes the proof of (2).

    We now fix an integral basis $\Delta=\{\alpha_1,\ldots,\alpha_l\}$ of $\Phi$ and we define $W_0$ to be the subgroup of $W$ generated by $\{w_{\alpha_1},\ldots,w_{\alpha_l}\}$. To prove (3), we first observe that $W$ acts on Weyl chambers, and we prove that $W_0$ acts transitively on them. A regular element $\mu$ lies in $C(\Delta)$ if and only if $(\mu,\alpha_i)$ for all $1\leq i\leq l$ and since $\Delta$ is a basis of $E$, it follows that the faces of $C(\Delta)$ are the hyperplanes $H_{\alpha_1},\ldots,H_{\alpha_l}$. Let $C$ is any Weyl chamber and let $\lambda\in C$, $\mu\in C(\Delta)$. Since the orbit of $\lambda$ under the action of $W_0$ is discrete, we may choose some $w\in W_0$ such that 
    $$|w(\lambda)-\mu|=\inf_{x\in W_0\lambda}|x-\mu|.$$
    If $w(\lambda)\not\in C(\Delta)$, then there is some $j\in\{1,\ldots,l\}$ such that $H_{\alpha_j}$ lies between $w(\lambda)$ and $\mu$. But $H_{\alpha_j}$ is the perpendicular bisector of $w(\lambda)$ and $w_{\alpha_j}w(\lambda)$, so 
    $$|w_{\alpha_j}w(\lambda)-\mu|<|w(\lambda)-\mu|,$$
    which is a contradiction. Thus, $w(\lambda)\in C(\Delta)$ and $w(C)\cap C(\Delta)\neq\emptyset$ so $w(C)=C(\Delta)$.

    Furthermore, since $w(\Delta)$ is an integral basis for any $w\in W$ and $w(C(\Delta))=C(w(\Delta))$, $W$ also acts transitively on the set of integral basis.

    Finally, we show that $W=W_0$. Fix some $\alpha\in\Phi$ and suppose there is some $w\in W_0$ such that $w(\alpha)\in\Delta$. Then $w_\alpha=w^{-1}w_{w(\alpha)}w\in W_0$. Thus it suffices to show that such $w\in W_0$ exists for all $\alpha\in\Phi$. To prove this, consider some $\gamma'\in H_\alpha$ but not lying in any $H_\beta$ for any $\beta\neq\pm\alpha$. Then choose some $\gamma$ sufficiently chose to $\gamma'$ so that $0<(\gamma,\alpha)<|(\gamma,\beta)|$ for any $\beta\neq\pm\alpha$. Then it is clear that $\alpha\in\Delta(\gamma)$ and since $W_0$ acts transitively on the integral basis, there exists some $w\in W_0$ such that $w(\Delta(\gamma))=\Delta$. In particular, $w(\alpha)\in\Delta$ as wished.
\end{proof}

It only remains to show that the action is simple; that is, if $w\in W$ satisfies $w(C(\Delta))=C(\Delta)$, then $w=1$. First, we need to introduce some vocabulary. The roots $\alpha_i\in\Delta$ are called \textit{simple roots} and the reflections $w_{\alpha_i}\in W$ are called \textit{simple reflections}. The \textit{height} of a root $\alpha=\sum_{i=1}^{l}c_i\alpha_i$ is defined as $\hht(\alpha)=\sum_{i=1}^{l}c_i$.

To simply notation, we denote $w_i$ for $w_{\alpha_i}$, and we remark that the set $\{w_1,\ldots,w_l\}$ is a \textit{minimal} set of generators of $W$. For each $w\in W$, we let $n(w)=\{\alpha\in\Phi:w(\alpha)\in\Phi^-\}=|\Phi^+\cap w^{-1}(\Phi^-)|$
and
$$l(w)=\min\{r\in\ZZ^{\geq0}:w=w_{i_1}\ldots w_{i_r} \text{ and all } w_{i_j} \text{ are simple reflections}\},$$
called the \textit{length} of the element $w$.

\begin{proposition}
    For all $w\in W$, we have $n(w)=l(w)$.
\end{proposition}

\begin{proof}[Proof of Theorem \ref{thm:weyl}(4)]
    Suppose that $w(C(\Delta))=C(\Delta)$. Then $w(\Delta)=\Delta$ and therefore $w(\Phi^+)=\Phi^+$. This implies that $l(w)=n(w)=0$ and therefore $w=1$ as required. This concludes the proof of Theorem \ref{thm:weyl}.
\end{proof}





\subsection{The Weight Lattice}

We begin this subsection with a basic result about simple reflections.

\begin{lemma}\label{lem:simplereflection}
    Let $\alpha_k\in\Delta\subset\Phi$ be a simple a root. Then $w_{\alpha_k}(\Phi^+\setminus\{\alpha_k\})=\Phi^+\setminus\{\alpha_k\}$.
\end{lemma}

\begin{proof}
    Let $\beta$ be a positive root other than $\alpha_k$. Then $\beta=\sum_{i=1}^{l}c_i\alpha_i$ and $c_j> 0$ for some $j\neq k$. Then the coefficient of $\alpha_j$ in $w_{\alpha_k}(\beta)=\beta-\check{\alpha_k}(\beta)\alpha_i\in\Phi$ is also $c_j>0$. Hence, $w_{\alpha_k}(\beta)\in\Phi^+\setminus\{\alpha_k\}$ since $\beta\neq-\alpha_k$.
\end{proof}

We now introduce the fundamental notion of a weight.

\begin{definition}
    Let $\Phi\subset E$ be a root system and let $\Delta=\{\alpha_1,\ldots,\alpha_l\}$ be an integral basis. Then the \textit{root lattice} $Q$ is $\ZZ\Phi=\oplus_{i=1}^l\ZZ\alpha_i$ and the \textit{weight lattice} is 
    $$P=\{\lambda\in E:\calpha(\lambda)\in\ZZ\text{ for all }\alpha\in\Phi\}.$$
    The elements of the weight lattice are called \textit{weights}.
\end{definition}

\begin{lemma}
    Let $\Delta=\{\alpha_1,\ldots,\alpha_l\}$ be an integral basis of a root system $\Phi\subset E$. Then $\check{\Delta}=\{\check{\alpha_1},\ldots,\check{\alpha_l}\}$ is an integral basis of $\cPhi\subset E^*$. Furthermore,
    $$P=\{\lambda\in E:\check{\alpha_i}(\lambda)\in\ZZ\text{ for all }1\leq i\leq l\}$$
\end{lemma}
\begin{proof}
    %Let $\alpha\in\Phi$ and assume first that $\alpha\in\Phi^+$, so $\alpha=\sum_{i=1}^{l}c_i\alpha_i$ for non-negative integers $c_1,\ldots,c_l$. Then 
    %$$\calpha=\frac{2\alpha^*}{(\alpha,\alpha)}=\sum_{i=1}^{l}\frac{2c_i}{(\alpha,\alpha)}\alpha_i^*=\sum_{i=1}^{l}\frac{(\alpha_i,\alpha_i)c_i}{(\alpha,\alpha)}\check{\alpha_i},$$
    %which implies that $\alpha$ is a $\QQ^{\geq0}$ linear combination of $\check{\Delta}$. 
    Let $\alpha\in\Phi^+$. We prove the result by induction over $\hht(\alpha)$. If $\hht(\alpha)=1$, then $\alpha$ is a simple root and the result is obvious. If $\hht(\alpha)\geq2$, we note that there is some $1\leq j\leq l$ such that $(\alpha,\alpha_j)>0$ as otherwise $\alpha\in -C(\Delta)$, a contradiction. Hence, $\check{\alpha}(\alpha_j)\in\ZZ^{>0}$, $w_{\alpha_j}(\alpha)\in\Phi^+$ and $\hht(w_{\alpha_j}(\alpha))<\hht(\alpha)$. By induction, $$\check{w_{\alpha_j}(\alpha)}=\sum_{i=1}^{l}c_i\check{\alpha_i}\text{ for some }c_i\in\ZZ^{\geq0}.$$
    On the other hand, $\check{w_{\alpha_j}(\alpha)}=w_{\check{\alpha_j}}(\calpha)=\calpha-\check{\check{\alpha_j}}(\calpha)\check{\alpha_j}=\calpha-\calpha(\alpha_j)\check{\alpha_j},$ so 
    $$\calpha=\calpha(\alpha_i)\check{\alpha_j}+\sum_{i=1}^{l}c_i\check{\alpha_i},$$
    proving the result. If $\alpha\in\Phi^-$, the argument is identical with $c_i\in\Phi^-$.

    Consequently, if $\lambda\in E$ satisfies that $\check{\alpha_i}(\lambda)\in\ZZ$ for all $1\leq i\leq l$, then $\calpha(\lambda)\in\ZZ$ for all $\alpha\in\Phi$. Thus, 
    $$P=\{\lambda\in E:\check{\alpha_i}(\lambda)\in\ZZ\text{ for all }1\leq i\leq l\},$$
    and the proof is complete
\end{proof}


\begin{definition}
    Let $\Phi$ be a root system with integral basis $\Delta=\{\alpha_1,\ldots,\alpha_l\}$. The elements $\{\omega_1,\ldots,\omega_l\}\subset E$ such that $\check{\alpha_i}(\omega_j)=\delta_{ij}$ for all $1\leq i,j\leq l$ are called the \textit{fundamental weights} of $\Phi$.
\end{definition}

It is clear from the definitions that $P=\oplus_{i=1}^l\ZZ\omega_i$. The fundamental weights satisfy the following identity.

\begin{lemma}
    Let $\delta=\frac{1}{2}\sum_{\alpha\in\Phi^+}\alpha$. Then for all simple roots $\alpha_k\in\Delta$, $w_{\alpha_k}(\delta)=\delta-\alpha_k$ and $\check{\alpha_k}(\delta)=1$. In particular, $\delta=\sum_{i=1}^{l}\omega_i$.
\end{lemma}
\begin{proof}
    This is a simple calculation using Lemma \ref{lem:simplereflection}. We note
    $$w_{\alpha_k}(\delta)=\frac{1}{2}+\sum_{\alpha\in\Phi^+\setminus\{\alpha_k\}}\frac{1}{2}w_{\alpha_k}(\alpha)=-\frac{1}{2}\alpha_k+\sum_{\alpha\in\Phi^+\setminus\{\alpha_k\}}\frac{1}{2}\alpha=\delta-\alpha_k,$$
    and this immediately implies that $\check{\alpha_k}(\delta)=1$. Hence, $\delta\in P$ and $\delta=\sum_{i=1}^{l}\omega_i$ as desired.
\end{proof}






    \newpage
    \section{Chevalley Groups: First Examples}
    The study of the structure of Lie algebras and their representation theory is a central tool in the representation theory of groups of Lie type. This is a consequence of the fact that the algebraic structure Lie algebra encapsulates to a large extent the interplay between the algebraic and topological properties of the group. To understand how one side helps understand the other, it is essential to give explicit methods that allows us to construct groups of Lie type from a Lie algebra and vice-versa. The next few chapters describe a way to construct a family of groups of Lie type from a Lie algebra $\fg$, denoted as \textit{adjoint groups of Lie type}. These arise as subgroups of the group $\Aut(\fg)$ of Lie automorphisms of $\fg$ and are constructed using the adjoint representation $\ad:\fg\to\mathrm{gl}(\fg)$.

This initial chapter presents explicitly three relatively simple examples with the aim to highlight and motivate some of the objects that appear in the general theory, which we will explain in later chapters. Before we start, however, we need a few general results on which the construction depend. 
\begin{definition}
    Let $\fg$ be a Lie algebra over a field $K$. A $K$-linear map $\delta:\fg\to\fg$ is called a \textit{derivation} if 
    $$\delta([xy])=[\delta(x)y]+[x\delta(y)]$$
    for all $x,y\in\fg$.
\end{definition}
Importantly, we note that for all $z\in\fg$, the map $\ad(z)$ is a derivation since $$\ad(z)([xy])=[z[xy]]=[[zx]y]+[x[zy]]=[\ad(z)(x),y]+[x,\ad(z)(y)],$$
where we have used the Jacobi identity. The main ingredient is the following:

\begin{proposition}
Let $\fg$ be a Lie algebra over a field $K$ of characteristic $0$ and let $\delta:\fg\to\fg$ be a nilpotent derivation with $\delta^n=0$ for $n\geq 0$. Then the map $$\exp(\delta)=I+\delta+\frac{\delta^2}{2}+\frac{\delta^3}{6}+\dots+\frac{\delta^{n-1}}{(n-1)!}:\fg\rightarrow\fg$$
is an automorphism of $\fg$ as a Lie algebra.
\end{proposition}
\begin{proof}
    The map $\exp(\delta)$ is certainly a $K$-linear map, with inverse given by $\exp(-\delta)$. It remains to show that it preserves the bracket. Since $\delta$ is a derivation, it is easy to check by induction that 
    $$\frac{\delta^r}{r!}([xy])=\sum_{\substack{i+j=r \\ i,j\geq 0}}\left[\frac{\delta^i}{i!}(x),\frac{\delta^{j}}{j!}(y)\right]$$ 
    for all $x,y\in\fg$. This immediately implies that
    $$\exp(\delta)([xy])=\sum_{r=0}^{\infty}\frac{\delta^r}{r!}([xy])=\sum_{r=0}^{\infty}\sum_{\substack{i+j=r \\ i,j\geq 0}}\left[\frac{\delta^i}{i!}(x),\frac{\delta^{j}}{j!}(y)\right]=\left[\sum_{i=0}^\infty\frac{\delta^i}{i!}(x),\sum_{j=0}^{\infty}\frac{\delta^j}{j!}\right]=[\exp(\delta)(x),\exp(\delta)(y)],$$
    as desired.
\end{proof}
\begin{cor}\label{cor:xalpha_automorphism}
    Let $\fg$ be a lie algebra over a field $K$ of characteristic $0$ and let $\alpha\in\Phi$ be a root of $\fg$. Then, for any $e_\alpha\in\fg_\alpha$ and $t\in K$, the map $x_\alpha(t):=\exp(\ad(te_\alpha))$ is an automorphism of $\fg$ as a Lie algebra.
\end{cor}

Finally, the following identities will be very helpful later on.
\begin{lemma}\label{lem:conjxalpla}
    Let $\fg$ be a Lie algebra over a field $K$ and let $\phi\in\Aut(\fg)$ be an automorphism of $\fg$. Then
    $$\phi\circ\exp(\ad(te_\alpha))\circ\phi^{-1}=\exp(\ad(t\phi(e_\alpha)))$$
    for all $e_\alpha\in\fg_\alpha$ and $t\in K$.
\end{lemma}

\begin{lemma}\label{lem:identityexp}
    Let $\fg\subseteq\mathrm{gl}_n(K)$ be a linear Lie algebra over a field $K$ and let $x\in\fg$ be nilpotent. Then 
    $$\exp(\ad(x))(y)=\exp(x)\circ y\circ \exp(x)^{-1}$$
    for all $y\in\fg$, where the composition is the standard multiplication of matrices in $\mathrm{gl}_n(K)$.
\end{lemma}

\subsection{Adjoint Chevalley groups of Type $A_1$}

Let $\fg_\CC=\mathrm{sl}_2(\CC)$ be the three-dimensional Lie algebra with basis given by
$$e=\begin{pmatrix}
    0 & 1\\
    0 & 0\\
\end{pmatrix},\quad h=\begin{pmatrix}
    1 & 0\\
    0 & -1\\
\end{pmatrix},\quad f=\begin{pmatrix}
    0 & 0\\
    1 & 0\\
\end{pmatrix},$$
and Lie bracket $[he]=2e$, $[hf]=-2f$ and $[ef]=h$. Thus, by choosing $\{e,h,f\}$ as basis of $\slii(\CC)$, the adjoint map $\mathrm{ad}:\slii(\CC)\to\mathrm{gl}(\slii(\CC))$ is given by 

$$e=\begin{pmatrix}
    0 & -2 & 0\\
    0 & 0 & 1\\
    0 & 0 & 0\\
\end{pmatrix},\quad h=\begin{pmatrix}
    2 & 0 & 0\\
    0 & 0 & 0\\
    0 & 0 & -2\\
\end{pmatrix},\quad f=\begin{pmatrix}
    0 & 0 & 0\\
    -1 & 0 & 0\\
    0 & 2 & 0\\
\end{pmatrix}.$$

We have that $\mathfrak{t}=\langle h\rangle$ is a Cartan subalgebra, so it acts on $\slii(\CC)$ by a semisimple endomorphism and we have the root space decomposition into eigenspaces 
$$\fg_\CC=\slii(\CC)=\langle h\rangle\oplus\langle e\rangle\oplus\langle f\rangle=\mathfrak{t}\oplus\fg_\alpha\oplus\fg_{-\alpha},$$
where $\alpha:\mathfrak{t}\to\CC$ satisfies $\alpha(h)=2$. The root system of $\slii(\CC)$ is therefore $\Phi=\{\alpha,-\alpha\}$, whose root lattice and weight lattice are $Q=\ZZ\alpha$ and $P=\ZZ\frac{\alpha}{2}$ respectively.

The crucial observation is that the basis $\{e,h,f\}$ is a Chevalley basis of $\fg_\CC=\slii(\CC)$, and therefore the $\ZZ$-module $\fg_\ZZ=\ZZ h\oplus\ZZ e\oplus\ZZ f$ is a Lie algebra over $\ZZ$. Hence, by fixing a field $K$, one obtains $\fg_K:=K\otimes_\ZZ\fg_\ZZ$, naturally a Lie algebra over $K$ with Lie bracket
$$[\mu_1\otimes x_1,\mu_2\otimes x_2]:=\mu_1\mu_2\otimes[x_1,x_2],$$
making it isomorphic to $\slii(K)$.

The adjoint Chevalley group of type $A_1$ over $K$ is defined as the subgroup of $G\leq\Aut(\fg_K)$ generated by $\{x_\alpha(t),x_{-\alpha}(t),h(\chi):t\in K\text{ and }\chi\in\Hom(\ZZ\alpha,K^*)\}$ where
$$x_\alpha(t)=\exp(\mathrm{ad}(te))=\begin{pmatrix}
    1 & -2t & -t^2\\
    0 & 1 & t\\
    0 & 0 & 1\\
\end{pmatrix},\quad x_{-\alpha}(t)=\exp(\mathrm{ad}(tf))=\begin{pmatrix}
    1 & 0 & 0\\
    -t & 1 & 0\\
    -t^2 & 2t & 1\\
\end{pmatrix},\quad h(\chi)=\begin{pmatrix}
    \chi(\alpha) & 0 & 0\\
    0 & 1 & 0\\
    0 & 0 & \chi(-\alpha)\\
\end{pmatrix}.$$

We know from Corollary \ref{cor:xalpha_automorphism} that $x_{\pm\alpha}(t)$ preserves the Lie bracket, and it is clear that $h(\chi)$ preserves it too. Moreover, since $\chi(-\alpha)=\chi(\alpha)^{-1}$, $G$ is also a subgroup of $\SL_3(K)$. To study the structure of $G$, we denote the \textit{root subgroups} $X_\alpha=\{x_\alpha(t):t\in K\}$ and $X_{-\alpha}=\{x_{-\alpha}(t):t\in K\}$, both isomorphic to $K$, and the \textit{diagonal subgroup} (or \textit{torus}) $H=\{h(\chi):\chi\in\Hom(\ZZ\alpha,K^*)\}$, isomorphic to $K^*$. Lemma \ref{lem:conjxalpla} implies that 
$$h(\chi)\circ x_\alpha(t)\circ h(\chi)^{-1}=x_\alpha(t\chi(\alpha))\quad\text{and}\quad h(\chi)\circ x_{-\alpha}(t)\circ h(\chi)^{-1}=x_{-\alpha}(t\chi(\alpha)^{-1}),$$ so $H$ normalizes $X_\alpha$ and $X_{-\alpha}$. Together with the fact that $H$ is abelian, we have that $\langle X_\alpha,X_{-\alpha}\rangle=G'\triangleleft G$ is the commutator subgroup of $G$. 

We would like a more concrete description of $G$ in terms of more familiar groups. The following proposition is a first step in this direction.

\begin{proposition}\label{prop:homSL2version1}
    There exists a homomorphism of groups 
    $\Psi':\SL_2(K)\rightarrow\langle X_\alpha,X_{-\alpha}\rangle=G'$
    such that 
    $$\Psi'\left(\begin{pmatrix}
        1 & t\\
        0 & 1\\
    \end{pmatrix}\right)=x_\alpha(t)\quad\text{and}\quad\Psi'\left(\begin{pmatrix}
        1 & 0\\
        t & 1\\
    \end{pmatrix}\right)=x_{-\alpha}(t)\quad\text{for all $t\in K$}.$$
\end{proposition}
\begin{proof}
    The map $\Psi'$ can be realized by an explicit representation of $\SL_2(K)$. Consider the action of $\SL_2(K)$ on the space of polynomials $K[x,y]$ given by 
    $$\begin{pmatrix}
        a & b\\
        c & d\\
    \end{pmatrix}\cdot f(x,y)=f(ax+cy,bx+dy),$$
    and restrict this action to the three dimensional subspace $K[x,y]_2$ of degree $2$ polynomials. By choosing the basis $\{-x^2,2xy,y^2\}$, one can easily check that the action is given by the homomorphism
    $$\begin{pmatrix}
        a& b\\
        c&d\\
    \end{pmatrix}\longmapsto\begin{pmatrix}
        a^2 & -2ab & -b^2\\
        -ac & ad+bc & bd\\
        -c^2 & 2cd & d^2\\
    \end{pmatrix},$$
    and this is precisely the desired homomorphism $\Psi'$.
\end{proof}

Moreover, one can easily check that $\begin{psmallmatrix}
    a&b\\
    c&d\\
\end{psmallmatrix}\in\ker\Psi'$ if and only if $b=c=0$ and $a=d=\pm{1}$. This implies
$$G'=\langle X_\alpha,X_{-\alpha}\rangle\cong\SL_2(K)/\{\pm I\}=\PSL_2(K),$$
and the bijection is explicitly given by $\Psi'$. In addition, this gives an explicit description of the torus $H':=H\cap G'$ of $G'$ since $\Psi'\left(\begin{psmallmatrix}
    a&b\\
    c&d\\
\end{psmallmatrix}\right)$ is of the form $h(\chi)$ if and only if $b=c=0$ and $a=d^{-1}$, in which case 
$$h_\alpha(\lambda)=\Psi'\left(\begin{pmatrix}
    \lambda & 0\\
    0 & \lambda^{-1}\\
\end{pmatrix}\right)=\begin{pmatrix}
    \lambda^2 & 0 & 0\\
    0 & 1 & 0\\
    0 & 0 & \lambda^{-2}\\
\end{pmatrix}.$$

This implies that $H'=H\cap G'$ contains the elements $h(\chi)$ such that $\chi\in\Hom(\ZZ\alpha,(K^*)^2)$. Equivalently, $h(\chi)\in H'$ if and only if there is some $\bar\chi\in\Hom(\ZZ\frac{\alpha}{2},K^*)\text{ such that }\bar\chi(\alpha)=\chi(\alpha)$. This is a general phenomenon, as we shall see later.

The next natural question, of course, is whether we have a global description of $G$ similar to $G'$. This is indeed the case, as we now show.

\begin{theorem}\label{thm:globalsl2}
    There exists a unique homomorphism of groups $\Psi:\GL_2(K)\to\langle X_\alpha,X_{-\alpha},H\rangle=G$
    extending $\Psi'$ such that 
    $$\Psi\left(\begin{pmatrix}
        s & 0\\
        0 & 1\\
    \end{pmatrix}\right)=\begin{pmatrix}
        s & 0 & 0\\
        0 & 1 & 0\\
        0 & 0 & s^{-1}\\
    \end{pmatrix}\in H$$
    for all $s\in K^*$.
\end{theorem}
\begin{proof}
    We claim that the natural extension of $\Psi'$ given by 
    $$\begin{pmatrix}
        a & b\\
        c & d\\
    \end{pmatrix}\in\GL_2(K)\longmapsto\frac{1}{ad-bc}\begin{pmatrix}
        a^2 & -2ab & -b^2\\
        -ac & ad+bc & bd\\
        -c^2 & 2cd & d^2\\
    \end{pmatrix}$$
    is the desired map $\Psi$. A simple but tedious calculation shows that this is a group homomorphism that coincides with $\Psi'$ on $\SL_2(K)$ and maps $\begin{psmallmatrix}
        s & 0\\
        0 & 1\\
    \end{psmallmatrix}$
    to $\mathrm{Diag}(s,1,s^{-1})$. Since $$\GL_2(K)=\left\langle\begin{pmatrix}
        1 & t\\
        0 & 1\\
    \end{pmatrix},\begin{pmatrix}
        1 & 0\\
        t & 1\\
    \end{pmatrix}, \begin{pmatrix}
        s & 0\\
        0 & 1\\
    \end{pmatrix}; t\in K, s\in K^*\right\rangle,$$
    it also follows that $\Ima\Psi=G$, as desired.
\end{proof}

It is an easy check that $\begin{psmallmatrix}
    a & b\\
    c & d\\
\end{psmallmatrix}\in\ker\Psi$ if and only if $b=c=0$ and $a=d$. Hence, $$G\cong\GL_2(K)/\{\lambda I,\lambda\in K^*\}=\PGL_2(K),$$
where the isomorphism is explicitly given by $\Psi$. Under this isomorphism, one can identify important subgroups of $G$ with subgroups of $\PGL_2(K)$. Indeed,

$$X_\alpha \longleftrightarrow\begin{pmatrix}
    1 & *\\
    0 & 1\\
\end{pmatrix},\quad X_{-\alpha} \longleftrightarrow\begin{pmatrix}
    1 & 0\\
    * & 1\\
\end{pmatrix},\quad H\longleftrightarrow\begin{pmatrix}
    * & 0\\
    0 & *\\
\end{pmatrix}.$$
Moreover, one also defines the \textit{monoidal} subgroup $N=\langle H,n_\alpha\rangle$, where $n_\alpha=\Psi_\alpha\begin{psmallmatrix}
    0 & 1\\
    -1 & 0\\
\end{psmallmatrix}$ and the \textit{Borel} subgroup $B=UH$. Under $\Psi$, these correspond to
$$N\longleftrightarrow\begin{pmatrix}
    * & 0\\
    0 & *\\
\end{pmatrix}\sqcup\begin{pmatrix}
    0 & *\\
    * & 0\\
\end{pmatrix},\quad B\longleftrightarrow\begin{pmatrix}
    * & *\\
    0 & *\\
\end{pmatrix}.$$

Of course, by intersecting these subgroups with $G'$, we get the same identifications inside $\PSL_2(K)\subseteq\PGL_2(K)$. Therefore, the following results also hold if we intersect the subgroups with $G'$.

Firstly, we note that $N$ is the normalizer of $H$ inside $G$. Therefore $H\triangleleft N$ and $N/H\cong C_2$, generated by $n_\alpha H$. Finally, a simple calculation shows the following.

\begin{theorem}[Bruhat Decomposition for $\slii$]
    Let $G,B,N$ be as above. Then 
    $$G=BNB=B\sqcup Bn_\alpha B.$$
    The same is true if we replace $G,B,N$ for $G',B',N'$.
\end{theorem}
\begin{proof}
    By using the identification given by $\Psi$, a simple calculation shows that 
    $$\begin{pmatrix}
        a_1 & b_1\\
        0 & d_1\\
    \end{pmatrix}\begin{pmatrix}
        0 & 1\\
        -1 & 0\\
    \end{pmatrix}\begin{pmatrix}
        a_2 & b_2\\
        0 & d_2\\
    \end{pmatrix}=\begin{pmatrix}
        -b_1a_2 & -b_1b_2+a_1d_2\\
        -d_1a_2 & -d_1b_2\\
    \end{pmatrix}.$$
    Since $d_1a_2\neq 0$, a simple calculation shows that $Bn_\alpha B=G\setminus B$, as desired.
\end{proof}





\subsection{Adjoint Chevalley groups of Type $A_2$}

In this section, we study the construction of Chevalley groups from the Lie algebra $\fg_\CC=\sliii(\CC)$, an $8$-dimensional Lie algebra with basis given by
$$\mathcal{B}=\{E_{11}-E_{22},E_{22}-E_{33},E_{12},E_{23},E_{13},E_{21},E_{32},E_{31}\}.$$
The diagonal matrices $\mathfrak{t}$ inside $\sliii(\CC)$ are a Cartan subalgebra of $\sliii(\CC)$ and are spanned by $h_1:=E_{11}-E_{22}$ and $h_2:=E_{22}-E_{33}$. The root space decomposition is 
$$\sliii(\CC)=\mathfrak{t}\oplus\langle E_{12}\rangle\oplus\langle E_{23}\rangle\oplus\langle E_{13}\rangle\oplus\langle E_{21}\rangle\oplus\langle E_{32}\rangle\oplus\langle E_{31}\rangle,$$
and a simple calculation shows that 
$$\langle E_{12}\rangle=\fg_{\alpha_1},\ \langle E_{23}\rangle=\fg_{\alpha_2},\ \langle E_{31}\rangle=\fg_{\alpha_1+\alpha_2},\ \langle E_{21}\rangle=\fg_{-\alpha_1},\ \langle E_{32}\rangle=\fg_{-\alpha_2},\ \langle E_{31}\rangle=\fg_{-\alpha_1-\alpha_2}$$
where $\alpha_1(h_{\alpha_1})=\alpha_2(h_{\alpha_2})=2$ and $\alpha_1(h_{\alpha_2})=\alpha_2(h_{\alpha_1})=-1$. Therefore, the root system of $\sliii(\CC)$ is $\Phi=\{\pm\alpha_1,\pm\alpha_2,\pm(\alpha_1+\alpha_2)\}$ of type $A_2$ and has root lattice $Q=\ZZ\alpha_1\oplus\ZZ\alpha_2$ and weight lattice $P=\ZZ\omega_1\oplus\ZZ\omega_2$ where $w_1=(2\alpha_1+\alpha_2)/3$ and $w_2=(\alpha_1+2\alpha_2)/3$.

Again, the basis $\mathcal{B}$ is a Chevalley basis, and for all $\alpha\in\Phi$, we let $e_\alpha=E_{ij}$ when $\langle E_{ij}\rangle=\fg_\alpha$. This implies that the $\ZZ$-module 
$$\fg_\ZZ=\ZZ h_{\alpha_1}\oplus\ZZ h_{\alpha_2}\oplus\sum_{\alpha\in\Phi}\ZZ e_\alpha$$
is a Lie algebra over $\ZZ$, and for any fixed field $K$, the vector space $\fg_K:=K\otimes_\ZZ\fg_\ZZ$ is a Lie algebra over $K$ isomorphic to $\sliii(K)$.
Similarly to the previous example, we define the \textit{adjoin Chevalley group of type $A_2$ over $K$} to be the subgroup of $\Aut(\fg_K)\cap\SL_8(K)$ generated by
$$\{x_\alpha(t),h(\chi):\alpha\in\Phi,t\in K,\chi\in\Hom(Q,K^*)\},$$ where $x_\alpha(t)=\exp(\ad(te_\alpha))$ and $h(\chi)$ satisfies $h(\chi)(t)=t$ for all $t\in\mathfrak{t}$ and $h(\chi)(e_\alpha)=\chi(\alpha)e_\alpha$ for all $\alpha\in\Phi$. 

We now study the structure of $G$ in an analogous way to the previous section. We define the \textit{root subgroups} $X_\alpha=\{x_\alpha(t):t\in K\}\cong K$ for each $\alpha$ and the \textit{diagonal subgroup} (or \textit{torus}) $H=\{h(\chi):\chi\in\Hom(Q,K^*)\}$. Moreover, we define the unipotent subgroups 
$$U=\langle X_{\alpha_1},X_{\alpha_2},X_{\alpha_1+\alpha_2}\rangle\quad\text{and}\quad V=\langle X_{-\alpha_1},X_{-\alpha_2},X_{-\alpha_1-\alpha_2}\rangle$$
and the Borel subgroup $B=UH=HU$.


Again, the torus $H$ normalizes each $X_\alpha$, so it normalizes $U$ and $V$. Hence, it follows that $\langle X_\alpha:\alpha\in\Phi\rangle=G'\triangleleft G$ is the commutator subgroup of $G$. The following Proposition, analogous to Proposition \ref{prop:homSL2version1} gives a `local' description of $G$.

\begin{proposition}
    For each $\alpha\in\Phi$, there exists an isomorphism of groups 
    $\Psi_\alpha:\SL_2(K)\rightarrow\langle X_\alpha,X_{-\alpha}\rangle\leq G'$
    such that 
    $$\Psi_\alpha\left(\begin{pmatrix}
        1 & t\\
        0 & 1\\
    \end{pmatrix}\right)=x_\alpha(t)\quad\text{and}\quad\Psi_\alpha\left(\begin{pmatrix}
        1 & 0\\
        t & 1\\
    \end{pmatrix}\right)=x_{-\alpha}(t)\quad\text{for all $t\in K$}.$$
\end{proposition}

\iffalse
\begin{proof}
    Again, the idea is to construct $\Psi_\alpha$ as a representation of $\SL_2(K)$. To do this, we break $$\fg_K=\langle h_\alpha\rangle^\perp\oplus(\langle h_\alpha\rangle\oplus\fg_\alpha\oplus\fg_{-\alpha})\oplus(\fg_) $$
\end{proof}
\fi
\vspace{0.2cm}
This result is very useful in practice, but we are interested in a `global' description of the group. To obtain such a description, we use Lemma \ref{lem:identityexp} to obtain the identity
$$x_\alpha(t)=\exp(te_\alpha)\circ y\circ\exp(te_\alpha)^{-1}$$
for all $y\in\sliii(K)$.
We note that for all $t\in K$,

\begin{equation}\label{eq:exp}
\exp(te_{\alpha_1})=\begin{pmatrix}
    1 & t & 0\\
    0 & 1 & 0\\
    0 & 0 & 1\\
\end{pmatrix},\quad\exp(te_{\alpha_2})=\begin{pmatrix}
    1 & 0 & 0\\
    0 & 1 & t\\
    0 & 0 & 1\\
\end{pmatrix},\quad\exp(te_{\alpha_1})=\begin{pmatrix}
    1 & 0 & t\\
    0 & 1 & 0\\
    0 & 0 & 1\\
\end{pmatrix},\quad\exp(te_{-\alpha})=\exp(te_\alpha)^T,
\end{equation}
and these matrices generate $SL_3(K)$. This gives a global description of $G'$.

\begin{theorem}
    There exists a surjective group homomorphism $\Psi':\SL_3(K)\to G'$ such that 
    $$\Psi'(A)(y)=AyA^{-1}$$
    for all $A\in\SL_3(K)$ and $y\in\sliii(K)$. Moreover, $\ker\Psi'$ is the scalar multiples of the identity in $\SL_3(K)$.
\end{theorem}
\begin{proof}
    From \eqref{eq:exp}, we know that $\Psi'(I+te_\alpha)=x_\alpha(t)$ for all $\alpha\in\Phi$ and $t\in K$. Since these matrices generate $\SL_3(K)$ and the maps $x_\alpha(t)$ generate $G'$, the map is a well-defined surjective group homomorphism. Finally, it is clear that $\ker\Psi'=\{A\in\SL_3(K):Ay=yA\text{ for all }y\in\sliii(K)\}=Z(\SL_3(K))$, which are the scalar multiples of the identity.
\end{proof}

Consequently, we get that $G'\cong\PSL_3(K)$, and under this isomorphism, we can identify subgroups of $G'$ with subgroups of $\PSL_3(K)$. Indeed,
\begin{equation*}
    X_{\alpha_1}\longleftrightarrow\begin{pmatrix}
        1 & * & 0\\
        0 & 1 & 0\\
        0 & 0 & 1\\
    \end{pmatrix},\quad X_{\alpha_2}\longleftrightarrow\begin{pmatrix}
        1 & 0 & 0\\
        0 & 1 & *\\
        0 & 0 & 1\\
    \end{pmatrix},\quad X_{\alpha_1+\alpha_2}\longleftrightarrow\begin{pmatrix}
        1 & 0 & *\\
        0 & 1 & 0\\
        0 & 0 & 1\\
    \end{pmatrix},\quad U\longleftrightarrow\begin{pmatrix}
        1 & * & *\\
        0 & 1 & *\\
        0 & 0 & 1\\
    \end{pmatrix}
\end{equation*}
and analogously for negative roots. Moreover, for any $A\in\SL_3(K)$, $\Psi'(A)$ preserves all root spaces if and only if $A$ is diagonal. Hence,

\begin{equation*}
    H'=H\cap G' \longleftrightarrow\begin{pmatrix}
        * & 0 & 0\\
        0 & * & 0\\
        0 & 0 & *\\
    \end{pmatrix},\quad B'=UH'\longleftrightarrow\begin{pmatrix}
        * & * & *\\
        0 & * & *\\
        0 & 0 & *\\
    \end{pmatrix}
\end{equation*}

These observations motivate two questions. Firstly, we would like to understand the torus $H'$ of $G'$ and, more importantly, we want a global description of $G$ extending our description of $G'$ similar to Theorem \ref{thm:globalsl2}. The first question can be answered by noting that, with respect to the Chevalley basis $\mathcal{B}$,
$$\Psi'(\mathrm{Diag}(\lambda,\lambda^{-1}\mu,\mu^{-1}))=\mathrm{Diag}(1,1,\lambda^2\mu^{-1},\lambda^{-1}\mu^2,\lambda\mu,\lambda^{-2}\mu,\lambda\mu^{-2},\lambda^{-1}\mu^{-1})=h(\chi),$$
where $\chi(\alpha_1)=\lambda^2\mu^{-1}$ and $\chi(\alpha_2)=\lambda^{-1}\mu^2$. Importantly, such a character can be extended to a character $\bar{\chi}$ of the weight lattice by setting $\bar\chi(\omega_1)=\lambda$ and $\bar\chi(\omega_2)=\mu$. By the construction above, this is also a sufficient condition, so have proved the following.

\begin{lemma}
    Let $\chi\in\Hom(Q,K^*)$ be a character of the root lattice. Then $h(\chi)\in H'=H\cap G'$ if and only if $\chi$ can be extended to a character of the weight lattice $P$.
\end{lemma}

Similarly, it is also possible to give an explicit global description of $G$ by extending $\Psi'$. 

\begin{theorem}
    There exists a unique surjective group homomorphism $\Psi:\GL_3(K)\to G$ extending $\Psi'$ such that 
    $$\Psi\left(\begin{pmatrix}
        s & 0 & 0\\
        0 & 1 & 0\\
        0 & 0 & 1\\
    \end{pmatrix}\right)=h(\chi_s),\quad\text{where }\chi_s(\alpha_1)=s\text{ and }\chi_s(\alpha_2)=1.$$
    In particular $h(\chi_s)\in H$ if and only if $s$ is a cube in $K^*$.
\end{theorem}

\begin{proof}
    To simplify notation, let $D_s\in\GL_3(K)$ be the diagonal matrix with $s$ at the $(1,1)$-entry and $1$ in the other two entries. Any matrix $A\in\GL_3(K)$ with $d=\det(A)$ can be expressed uniquely as $A=D_dA'$ where $A'\in\SL_3(K)$, so we might define 
    $$\Psi(A)=h(\chi)\Psi'(A').$$
    Of course, we now need to show that $\Psi$ is a group homomorphism. By Lemma \ref{lem:conjxalpla}, we know that 
    $$h(\chi_s)x_{\alpha_1}(t)=x_{\alpha_1}(st)h(\chi_s)\quad\text{and}\quad h(\chi_s)x_{\alpha_2}(t)=x_{\alpha_2}(t)h(\chi_s),$$
    which agrees with the identities
    $$\begin{pmatrix}
        s & 0 & 0\\
        0 & 1 & 0\\
        0 & 0 & 1\\
    \end{pmatrix}\begin{pmatrix}
        1 & t & 0\\
        0 & 1 & 0\\
        0 & 0 & 1\\
    \end{pmatrix}=\begin{pmatrix}
        1 & st & 0\\
        0 & 1 & 0\\
        0 & 0 & 1\\
    \end{pmatrix}\begin{pmatrix}
        s & 0 & 0\\
        0 & 1 & 0\\
        0 & 0 & 1\\
    \end{pmatrix}\text{ and }\begin{pmatrix}
        s & 0 & 0\\
        0 & 1 & 0\\
        0 & 0 & 1\\
    \end{pmatrix}\begin{pmatrix}
        1 & 0 & 0\\
        0 & 1 & t\\
        0 & 0 & 1\\
    \end{pmatrix}=\begin{pmatrix}
        1 & 0 & 0\\
        0 & 1 & t\\
        0 & 0 & 1\\
    \end{pmatrix}\begin{pmatrix}
        s & 0 & 0\\
        0 & 1 & 0\\
        0 & 0 & 1\\
    \end{pmatrix}.$$
    The same compatibility is true for any $\alpha\in\Phi$, $t\in K$ and $s\in K^*$. Since $\SL_3(K)$ is generated by $\{x_\alpha(t):\alpha\in\Phi, t\in K\}$, we have that for any $s,u\in K^*$ and $A,B\in\SL_3(K)$,
    \begin{align*}
        \Psi(D_sAD_uB)=\Psi(D_sD_uD_u^{-1}AD_uB)=h(\chi_{s})h(\chi_u)\Psi'(D_u^{-1}AD_u)\Psi'(B)=\\
        h(\chi_s)\Psi'(D_u(D_u^{-1}AD_u)D_u^{-1})h(\chi_u)\Psi'(B)=h(\chi_s)\Psi'(A)h(\chi_u)\Psi'(B)=\Psi(D_sA)\Psi(D_uB).
    \end{align*}
    This concludes the proof.
\end{proof}

Since $G$ is generated by $G'$ and $H$, it is still the case that for $A\in\GL_3$, $\Psi(A)$ is diagonal if and only if $A$ is diagonal. In particular, by tracing through the construction of $\Psi$, one can easily show that 
$$\Psi(\mathrm{Diag}(s,u,v))=h(\chi),\quad\text{where }\chi(\alpha_1)=su^{-1}\text{ and }\chi(\alpha_2)=uv^{-1}.$$
Therefore, $\ker\Psi=\{\lambda I:\lambda\in K^*\}$ are the scalar multiples of the identity, so $G\cong\PGL_3(K)$. Moreover, the identification given by $\Psi$ of the subgroups of $G$ with subgroups of $\PGL_2(K)$ is analogous to the identification given by $\Psi'$.

We are finally ready to give the Bruhat decomposition of $G$ (and $G'$). For each $\alpha\in\Phi$, let $n_\alpha=\Psi_\alpha\left(\begin{psmallmatrix}
    0 & 1\\
    -1 & 0\\
\end{psmallmatrix}\right)$, and define the monoidal subgroups
$$N=\langle H,n_\alpha:\alpha\in\Phi\rangle,\quad N'=N\cap G'=\langle H',n_\alpha:\alpha\in\Phi\rangle.$$
Under the isomorphism given by $\Psi$, we have that 
\begin{equation*}
    n_{\alpha_1}\longleftrightarrow\begin{pmatrix}
        0 & 1 & 0\\
        -1 & 0 & 0\\
        0 & 0 & 1\\
    \end{pmatrix},\quad n_{\alpha_2}\longleftrightarrow\begin{pmatrix}
        1 & 0 & 0\\
        0 & 0 & 1\\
        0 & -1 & 0\\
    \end{pmatrix},\quad n_{\alpha_1+\alpha_2}\longleftrightarrow\begin{pmatrix}
        0 & 0 & 1\\
        0 & 1 & 0\\
        -1 & 0 & 0\\
    \end{pmatrix}
\end{equation*}
and therefore 
\begin{equation*}
    N\longleftrightarrow\begin{pmatrix}
        * & 0 & 0\\
        0 & * & 0\\
        0 & 0 & *\\
    \end{pmatrix}\sqcup\begin{pmatrix}
        0 & * & 0\\
        * & 0 & 0\\
        0 & 0 & *\\
    \end{pmatrix}\sqcup\begin{pmatrix}
        * & 0 & 0\\
        0 & 0 & *\\
        0 & * & 0\\
    \end{pmatrix}\sqcup\begin{pmatrix}
        0 & 0 & *\\
        * & 0 & 0\\
        0 & * & 0\\
    \end{pmatrix}\sqcup\begin{pmatrix}
        0 & * & 0\\
        0 & 0 & *\\
        * & 0 & 0\\
    \end{pmatrix}\sqcup\begin{pmatrix}
        0 & 0 & *\\
        0 & * & 0\\
        * & 0 & 0\\
    \end{pmatrix}
\end{equation*}
From this perspective, it is clear that that $N$ is in fact the normalizer of $H=B\cap N$ inside $G$, and that the quotient $N/H$ is isomorphic to $S_3$, generated by the elements $n_{\alpha_1}H$ and $n_{\alpha_2}H$. Now the proof of the Bruhat decomposition for $G$ is a tedious analogous to Theorem \ref{thm:bruhatsl2}.

\begin{theorem}[Bruhat decomposition for $\sliii$] 
    Let $G,B,N$ as above. Then 
    $$G=BNB=B\sqcup Bn_{\alpha_1}B\sqcup Bn_{\alpha_2}B\sqcup Bn_{\alpha_1}n_{\alpha_2}B\sqcup Bn_{\alpha_2}n_{\alpha_1}B\sqcup Bn_{\alpha_1}n_{\alpha_2}n_{\alpha_1}B.$$  
    The same is true if we replace $G,B,N$ for $G',B',N'$.
\end{theorem}







\subsection{Adjoint Chevalley groups of Type $A_n$}

Having explicitly described the structure for adjoint Chevalley groups of Type $A_2$, we are ready to describe the general structure of Chevalley groups of type $A_n$.

Consider the complex Lie algebra $\fg_\CC=\mathrm{sl}_{n+1}(\CC)$ with Cartan subalgebra $\mathfrak{t}$ given by diagonal matrices and spanned by $h_i:=E_{i,i}-E_{i+1,i+1}$ for $1\leq i\leq n$. The root space decomposition 
$$\fg_\CC=\mathrm{sl}_n(\CC)=\mathfrak{t}\oplus\sum_{\alpha\in\Phi}\fg_{\alpha},$$
where $\Phi$ is a root system of type $A_n$. We may choose simple roots $\alpha_1,\ldots,\alpha_n$ such that $\fg_{\alpha_i}=\langle E_{i,i+1}\rangle$ for $1\leq i\leq n$ and the other positive roots of $\Phi$ are given by $\alpha=\alpha_i+\alpha_{i+1}+\cdots+\alpha_{j-1}$ for $1\leq i< j\leq n$ and they satisfy $\fg_\alpha=\langle E_{i,j}\rangle$ and $\fg_{-\alpha}=\langle E_{j,i}\rangle$. To simply notation, given $i<j$ and $\alpha$ as above, we let $e_\alpha=E_{i,j}$ and $e_{-\alpha}=E_{j,i}$.
We remark that 
$$\alpha_i(h_{\alpha_j})=\begin{cases}
    2 & \text{ if } i=j, \\
    -1 & \text{ if } |i-j|=1,\\
    0 & \text{ otherwise,} 
\end{cases}$$

and therefore the fundamental weights are given by 
\begin{equation}
    \begin{pmatrix}
        \omega_1\\
        \vdots\\
        \vdots\\
        \vdots\\
        \omega_n
    \end{pmatrix}=\begin{pmatrix}
        2 & -1 & & & \\
        -1 & 2 & -1 & &\\
         & -1 & 2 & \ddots & \\
         & & \ddots & \ddots & -1\\
         & & & -1 & 2\\ 
    \end{pmatrix}^{-1}
    \begin{pmatrix}
        \alpha_1\\
        \vdots\\
        \vdots\\
        \vdots\\
        \alpha_n\\
    \end{pmatrix}.
\end{equation}

Since the change of basis matrix above has determinant $n+1$, it follows that the root lattice $Q=\sum_{i=1}^n\ZZ\alpha_i$ has index $n+1$ inside the weight lattice $P=\sum_{i=1}^n\omega_i$.

In this general setting, the basis 
$$\mathcal{B}=\{h_1,\ldots,h_n,e_\alpha:\alpha\in\Phi\}$$
is a Chevalley basis, so the $\ZZ$-module
$$\fg_\ZZ=\sum_{i=1}^{l}\ZZ h_i\oplus\sum_{\alpha\in\Phi}\ZZ e_\alpha$$
is a Lie algebra over $\ZZ$. If we fix any field $K$, the $K$-vector space $\fg_K=K\otimes_\ZZ\fg_\ZZ$ is a Lie algebra over $K$. For each $t\in K$, the map  




    \newpage
    \section{Construction of Chevalley Groups}
    \input{construction_chevalley.tex}

    \section{Structure of Chevalley Groups}
    \input{structure_chevalley.tex}

    \section{Classical Bruhat Decomposition}
    \input{classical_bruhat.tex}

    

    \newpage
    \section{The Affine Weyl Group}
    Throughout this section, we fix a root system $\Phi$ on an Euclidean space $E$ with integral basis $\Delta=\{\alpha_1,\ldots,\alpha_l\}$ and Weyl group $W$. In Chapter \ref{chp:weylgroup}, we introduced two full rank lattices of $E$; the root lattice $Q=\oplus_{i=1}^l\ZZ\alpha_i$ and the weight lattice $P=\oplus_{i=1}^l\ZZ\omega_i$, which contains the root lattice $Q$. We now construct the dual lattices in $E^*$.

\begin{definition}
    Let $\Phi\subset E$ be a root system with root lattice $Q$ and weight lattice $P$. Then the \textit{dual root lattice} is 
    $$Q^\perp=\{x\in E^*:\langle\lambda,x\rangle\in\ZZ\text{ for all }\lambda\in Q\}\subset E^*$$
    and the \textit{dual weight lattice} is 
    $$P^\perp=\{x\in E^*:\langle\lambda,x\rangle\in\ZZ\text{ for all }\lambda\in P\}\subset E^*.$$
\end{definition}

The following lemma shows that the dual lattices have desirable properties with respect to the dual root system $\cPhi$. Recall that $\check{\Delta}=\{\check{\alpha_1},\ldots,\check{\alpha_l}\}$ is an integral basis of $\cPhi$.

\begin{lemma}
    The dual weight lattice is the root lattice of $\cPhi$ and the dual root lattice is the weight lattice of $\cPhi$. That is,
    $$P^\perp=\bigoplus_{i=1}^l\ZZ\check{\alpha_i}\quad\text{and}\quad Q^\perp=\bigoplus_{i=1^l}\ZZ\epsilon_i^*,$$
    where $\langle\epsilon_j^*,\check{\check{\alpha_i}}\rangle=\delta_{ij}$ for all $1\leq i,j\leq l$.
\end{lemma}
\begin{proof}
    The first statement follows directly from the fact that $P=\oplus_{i=1}^l\ZZ\omega_i$ and that $\langle\omega_j,\check{\alpha_j}\rangle=\delta_{ij}$. The second statement follows from the fact that $Q=\oplus_{i=1}^l\ZZ\alpha_i$ and $\langle\alpha_i,\epsilon_j^*\rangle=\langle\epsilon_j^*,\check{\check{\alpha_i}}\rangle=\delta_{ij}$.
\end{proof}

For each $\alpha\in\Phi$ and $k\in\ZZ$, define the hyperplane
$$H_{\alpha,k}=\{x\in E^*:\langle\alpha,x\rangle=k\}\subset E^*$$
and we consider the reflections along each $H_{\alpha,k}$ given by
$$w_{\alpha,k}(x)=x-\langle\alpha,x\rangle\calpha+k\calpha,\quad x\in E^*.$$
\begin{definition}
    The \textit{affine Weyl group} of $\Phi$, denoted by $W_a$, is the group generated by all $\{w_{\alpha,k}:\alpha\in\Phi, k\in\ZZ\}$.
\end{definition}

Since, $w_{\alpha,0}=w_{\check{\alpha}}$, it follows that $W_\Phi\cong W_{\cPhi}$ is naturally a subgroup of $W_a$. Given $d\in Q^\perp$, we consider the translation map $T(d):x\mapsto x+d,\ d\in E^*$ and we define the abelian groups 
$$D=\{T(d):d\in Q^\perp\}\quad\text{and}\quad D'=\{T(d):d\in P^\perp\}.$$
It is immediate by the definitions that $$w_{\alpha,k}=T(k\calpha)\circ w_{\alpha,0}\quad\text{and}\quad w\circ T(d)\circ w^{-1}=T(w(d))\quad \text{for all}\quad\alpha\in\Phi, k\in\ZZ, d\in Q^\perp\text{ and }w\in W.$$
Moreover, $D\cap W=D'\cap W={1}$ and hence $W_a=D'\rtimes W$.

\begin{definition}
    The \textit{extended affine Weyl group} is the group $\eW:=D\rtimes W$.
\end{definition}

We remark that $W_a=D'W$ is a normal subgroup of $\eW$. Indeed, if $T(d)\in D$, then $$T(d)w_{\alpha,k}T(-d)=T(d+k\calpha) w_{\alpha,0}T(-d)=T(d+k\calpha-w_{\alpha,0}(d))w_{\alpha,0}=T((k+\langle\alpha,d\rangle)\calpha)w_{\alpha,0}=w_{\alpha,k+\langle\alpha,d\rangle}\in W_a$$
since $d\in Q^\perp$ so $\langle\alpha,d\rangle\in\ZZ$.

\iffalse
\begin{lemma}\label{lem:normalweyl}
    The affine Weyl group $W_a$ is a normal subgroup of the extended affine Weyl group $\eW$. Moreover,
    $$\eW/W\cong D/D'\cong Q^\perp/P^\perp\cong P/Q$$
    is the fundamental group associated to $\Phi$. 
\end{lemma}

\begin{proof}
    Firstly, we note that given any $\sigma=T(d)w\in DW$, we can recover $d\in Q^\perp$ and $w\in W$ by $d=\sigma(0)$ and $w=\sigma\circ(T(-d))$. Then, let $T(d)w\in DW$ and $T(d')w'\in D'W$ for $d\in Q^\perp$ and $d'\in P^\perp$. By the initial observation, $T(d)wT(d')(T(d)w)^{-1}\in D'W$ if and only if
    $$T(d)wT(d')(T(d)w)^{-1}(0)=ww'w^{-1}(-d)+w(d')+d\in P^\perp.$$
    Since $d'\in P^\perp$ and $W$ acts on $P^\perp$, the result follows immediately from the following lemma:
    \begin{lemma}
        Let $d\in Q^\perp$ and $w\in W$. Then $d-w(d)\in P^\perp$.
    \end{lemma}
    \begin{proof}
        Since $Q=\oplus_{i=1}^l\ZZ\epsilon_i^*$ and $w\in W$ is linear, it is enough to prove the result for $d=\epsilon_k^*,1\leq k\leq l$. Furthermore, since $W$ is generated by $\{w_{1},\ldots,w_{l}\}$, where $w_i=w_{\check{\alpha_i}}$, we may write $w=w_{i_1}\cdots w_{i_l}$ as a product of simple reflections. We can express $\epsilon_k^*-w(\epsilon_k^*)$ as a telescopic sum
        $$\epsilon_k^*-w(\epsilon_k^*)=\sum_{j=1}^r w^{(j)}(\epsilon_k^*-w_{i_j}\epsilon_k^*)\text{ where } w^{(j)}=w_{i_1}\cdots w_{i_{j-1}}.$$
        By definition, we have that $\epsilon_k^*-w_{i_j}(\epsilon_k^*)=\langle\epsilon_k^*,\check{\check{\alpha_{i_j}}}\rangle\alpha_{i_j}\in P^\perp$, and since $W$ acts on  $P^\perp$ the result follows.        
    \end{proof}
    This concludes the proof of Lemma \ref{lem:normalweyl}
\end{proof}
\fi

Analogously to the standard Weyl group, we consider the connected components of $E^*\setminus\bigcup_{\substack{\alpha\in\Phi\\k\in\ZZ}}H_{\alpha,k}$, called \textit{alcoves}. We remark that $\eW=D\rtimes W$ acts on the set of alcoves. Furthermore, the set $A(\Delta)=\{x\in E:\langle\alpha,x\rangle\in(0,1)\text{ for all }\alpha\in\Phi^+\}\subset C(\Delta)$ is called the \textit{fundamental alcove} and it can also be described as 
$$A(\Delta)=\{x\in E:\langle\alpha_i,x\rangle>0\text{ for all }1\leq i\leq l\text{ and }\langle\alpha_0,x\rangle<1\}.$$
Hence the faces of the fundamental alcove $A(\Delta)$ (called \textit{facets}) are $H_{\alpha_0,1},H_{\alpha_1,0},\ldots,H_{\alpha_l,0}$. To simplify notation, we let $H_0=H_{\alpha_0,1}$ and $H_i=H_{\alpha_i,0}$ for $1\leq i\leq l$. Similarly, we let $w_i=w_{\alpha_i,0}$ and $w_0=w_{\alpha_0,1}=T(\check{\alpha_0})\circ w_{\alpha_0,0}$.

The following structure theorem is the analogous version of Theorem \ref{thm:weyl} for the affine Weyl group.

\begin{theorem}\label{thm:affineweyl}
    The affine Weyl group $W_a=D'\rtimes W$ is generated by $\{w_0,w_1,\ldots,w_l\}$ and this is a minimal set of generators. Moreover, $W_a$ acts simply transitively on the set of alcoves.
\end{theorem}
Similarly to the proof of Theorem \ref{thm:weyl}, the hardest fact to prove is that the action on the alcoves is simple.

\begin{proof}
    We proceed in a similar fashion to Theorem \ref{thm:weyl}. Consider the subgroup $W_{a,0}$ of $W_a$ generated by $\{w_0,w_1,\ldots,w_l\}$ and fix an alcove $A$. Choose some $x\in A$ and $y\in A(\Delta)$. Since the orbit of $x$ under $W_{a,0}$ is discrete, we may choose some $w\in W_{a,0}$ so that 
    $|w(x)-y|=\inf_{z\in W_{a,0}x}|z-y|$. If $w(x)\not\in A(\Delta)$, then there is some hyperplane $H_i, 0\leq i\leq l$ between $w(x)$ and $y$. But $H_i$ is the perpendicular bisector of $w_iw(x)$ and $w(x)$, and this implies that
    $$|w_iw(x)-y|<|w(x)-y|,$$
    a contradiction since $w_iw\in W_{a,0}$. Thus, $w(A)\cap A(\Delta)\neq\emptyset$, so $w(A)=A(\Delta)$ and this shows that the action of $W_{a,0}$ on the set of alcoves is transitive.

    Now fix some $\alpha\in\Phi$ and $k\in\ZZ$ and consider an alcove $A$ such that $H_{\alpha,k}$ is a facet of $A$. Choose some $w\in W_{a,0}$ such that $w(A)=A(\Delta)$. Then $w(H_{\alpha_k})=H_i$ for some $0\leq i\leq l$ and therefore $w_{\alpha,k}=w^{-1}w_iw\in W_{a,0}$. This proves that $W_a=W_{a,0}$ is generated by $\{w_0,w_1,\ldots,w_l\}$ as desired. To prove that the action is simple, we need a few preparations.
\end{proof}

Analogously to the Weyl group, given $w\in W_a$, we define the \textit{length} of $w$ to be 
$$l(w)=\min\{r\in\ZZ^{\geq0}:w=w_{i_1}\cdots w_{i_r}\text{ and all }w_{i_j}\in\{w_0,w_1,\ldots,w_l\}\}.$$
In this setting, we also need a generalization of the function $n(w)=|\cPhi^+\cap w^{-1}(\cPhi^-)|=|\cPhi^+\cap w(\cPhi^-)|,\ w\in W$ for elements of $W_a$. To that aim given a hyperplane $H_{\alpha,k}$ and alcoves $A_1$ and $A_2$, we say that $A_1\sim A_2\ (H_{\alpha,k})$ if $A_1$ and $A_2$ are in the same side of $H_{\alpha,k}$ and $A_1\nsim A_2\ (H_{\alpha,k})$ otherwise. Given $\sigma\in \eW$, we define 
$$\mathcal{H}(\sigma)=\{H_{\alpha,k}:\alpha\in\Phi,k\in\ZZ\text{ and }A(\Delta)\nsim \sigma(A(\Delta))\ (H_{\alpha,k})\}.$$
We note that this is a finite set, so we define $n(\sigma)=|\mathcal{H}(\sigma)|$. This indeed generalizes $n(w)$ for $w\in W$ since for $\calpha\in\cPhi^+$, we have that $H_{\alpha,k}\in\mathcal{H}(w)$ if and only if $k=0$ and $w^{-1}(\calpha)\in\cPhi^-$. The proof that the action of $W_a$ is simple is a direct consequence of the following proposition.

\begin{proposition}
    Let $\sigma\in W_a$ with $l(\sigma)=r$ and suppose that $\sigma=w_{i_1}\cdots w_{i_r}$ is a reduced expression. Then
    $$\mathcal{H}(\sigma)=\{H_{i_1},w_{i_1}(H_{i_2}),w_{i_1}w_{i_2}(H_{i_3}),\ldots,w_{i_1}\cdots w_{i_{r-1}}(H_{i_r})\},$$
    and all these hyperplanes are distinct. In particular, $n(\sigma)=r=l(w)$.
\end{proposition}

We are now ready to finish the proof of Theorem \ref{thm:affineweyl}

\begin{proof}[Proof of Theorem \ref{thm:affineweyl}]
    We already know that the action of $W_a$ on the set of alcoves is transitive. If $\sigma\in W_a$ satisfies that $\sigma(A(\Delta))=A(\Delta)$, then $\mathcal{H}(\sigma)=\emptyset$ and $l(\sigma)=n(\sigma)=|\mathcal{H}(\sigma)|=0$. Therefore, $\sigma=1$ and the proof is complete.
\end{proof}
































\newpage
\subsection{Examples}
\vspace{0.5cm}
\definecolor{ffqqqq}{rgb}{1,0,0}
\definecolor{ududff}{rgb}{0.30196078431372547,0.30196078431372547,1}
\definecolor{uuuuuu}{rgb}{0.26666666666666666,0.26666666666666666,0.26666666666666666}
\begin{center}
    \begin{tikzpicture}[line cap=round,line join=round,>=triangle 45,x=2.5cm,y=2.5cm]
        \clip(-1.3,-1.3) rectangle (1.3,1.5);
        \draw [line width=0.4pt,dash pattern=on 2pt off 2pt] (0,-1.7678450876512102) -- (0,2.2749034653178044);
        \draw [line width=0.4pt,dash pattern=on 2pt off 2pt,domain=-2.599612871193306:3.164564097878663] plot(\x,{(-0--0.5*\x)/0.8660254037844386});
        \draw [line width=0.4pt,dash pattern=on 2pt off 2pt] (0.5,-1.7678450876512102) -- (0.5,2.2749034653178044);
        \draw [line width=0.4pt,dash pattern=on 2pt off 2pt,domain=-2.599612871193306:3.164564097878663] plot(\x,{(--0.28867513459481287--0.28867513459481287*\x)/0.5});
        \begin{scriptsize}
            \draw [fill=black] (1,0) circle (1.2pt);
            \draw[color=black] (1.0519019508432352,0.08725485318699086) node {$\alpha_1$};
            \draw [fill=black] (-0.5,0.8660254037844386) circle (1.2pt);
            \draw[color=black] (-0.4478273510646299,0.9544896234206665) node {$\alpha_2$};
            \draw [fill=uuuuuu] (0.5,0.8660254037844386) circle (1.2pt);
            \draw[color=uuuuuu] (0.8,0.8) node {$\alpha_1+\alpha_2$};
            \draw [fill=uuuuuu] (-1,0) circle (1.2pt);
            \draw[color=uuuuuu] (-1.05,0.1) node {$-\alpha_1$};
            \draw [fill=uuuuuu] (0.5,-0.8660254037844386) circle (1.2pt);
            \draw[color=uuuuuu] (0.5498186628132109,-0.7799799170466849) node {$-\alpha_2$};
            \draw [fill=uuuuuu] (-0.5,-0.8660254037844386) circle (1.2pt);
            \draw[color=uuuuuu] (-0.5,-0.7799799170466849) node {$-\alpha_1-\alpha_2$};
            \draw [fill=ududff] (0,0) circle (1.2pt);
            \draw[color=ududff] (0.06,-0.06) node {$0$};
            \draw[color=black] (-0.35,1.4) node {$\langle\cdot,\check{\alpha_1}\rangle=0$};
            \draw[color=black] (-0.9,-0.2) node {$\langle\cdot,\check{\alpha_2}\rangle=1$};
            \draw [fill=ffqqqq] (0.5,0.28867513459481287) circle (1.2pt);
            \draw[color=ffqqqq] (0.65,0.25) node {$\omega_1$};
            \draw [fill=ffqqqq] (0,0.5773502691896257) circle (1.2pt);
            \draw[color=ffqqqq] (-0.1,0.6675848874035106) node {$\omega_2$};
            \draw[color=black] (0.85,1.4) node {$\langle\cdot,\check{\alpha_1}\rangle=1$};
            \draw[color=black] (-0.6,-0.6) node {$\langle\cdot,\check{\alpha_2}\rangle=0$};
        \end{scriptsize}
    \end{tikzpicture}
\end{center}

\vspace{0.5cm}

\begin{center}
    \begin{tikzpicture}[line cap=round,line join=round,>=triangle 45,x=2.5cm,y=2.5cm]
        \clip(-1.3,-1.3) rectangle (1.3,1.5);
        \draw [line width=0.4pt,dash pattern=on 2pt off 2pt] (0,-2) -- (0,2);
        \draw [line width=0.4pt,dash pattern=on 2pt off 2pt,domain=-2.599612871193306:3.164564097878663] plot(\x,{(-0--0.5*\x)/0.8660254037844386});
        \draw [line width=0.4pt,dash pattern=on 2pt off 2pt] (0.5,-2) -- (0.5,2);
        \draw [line width=0.4pt,dash pattern=on 2pt off 2pt,domain=-2.599612871193306:3.164564097878663] plot(\x,{(--0.28867513459481287--0.28867513459481287*\x)/0.5});
        \draw [line width=0.4pt,dash pattern=on 2pt off 2pt] (-2,1.732) -- (2,-0.5773);
        \begin{scriptsize}
            \draw [fill=black] (1,0) circle (1.2pt);
            \draw[color=black] (1.0519019508432352,0.08725485318699086) node {$\check{\alpha_1}$};
            \draw [fill=black] (-0.5,0.8660254037844386) circle (1.2pt);
            \draw[color=black] (-0.4478273510646299,0.9544896234206665) node {$\check{\alpha_2}$};
            \draw [fill=uuuuuu] (0.5,0.8660254037844386) circle (1.2pt);
            \draw[color=uuuuuu] (0.8,0.8) node {$\check{\alpha_1}+\check{\alpha_2}$};
            \draw [fill=uuuuuu] (-1,0) circle (1.2pt);
            \draw[color=uuuuuu] (-1.05,0.1) node {$-\check{\alpha_1}$};
            \draw [fill=uuuuuu] (0.5,-0.8660254037844386) circle (1.2pt);
            \draw[color=uuuuuu] (0.5498186628132109,-0.7799799170466849) node {$-\check{\alpha_2}$};
            \draw [fill=uuuuuu] (-0.5,-0.8660254037844386) circle (1.2pt);
            \draw[color=uuuuuu] (-0.5,-0.7799799170466849) node {$-\check{\alpha_1}-\check{\alpha_2}$};
            \draw [fill=ududff] (0,0) circle (1.2pt);
            \draw[color=ududff] (0.06,-0.06) node {$0$};
            \draw[color=black] (-0.35,1.4) node {$\langle{\alpha_1},\cdot\rangle=0$};
            \draw[color=black] (-0.9,-0.2) node {$\langle{\alpha_2},\cdot\rangle=1$};
            \draw [fill=ffqqqq] (0.5,0.28867513459481287) circle (1.2pt);
            \draw[color=ffqqqq] (0.68,0.285) node {$\epsilon_1^*$};
            \draw [fill=ffqqqq] (0,0.5773502691896257) circle (1.2pt);
            \draw[color=ffqqqq] (-0.1,0.73) node {$\epsilon_2^*$};
            \draw[color=black] (0.85,1.4) node {$\langle{\alpha_1},\cdot\rangle=1$};
            \draw[color=black] (-0.6,-0.6) node {$\langle{\alpha_2},\cdot\rangle=0$};
            \draw[color=black] (-1,0.87) node {$\langle{\alpha_0},\cdot\rangle=1$};
        \end{scriptsize}
    \end{tikzpicture}
\end{center}

\vspace{0.5cm}

\begin{center}
    \begin{tikzpicture}[line cap=round,line join=round,>=triangle 45,x=2.5cm,y=2.5cm]
        \clip(-1.5,-1.5) rectangle (1.5,1.6);
        \draw [line width=0.4pt,dash pattern=on 2pt off 2pt] (0,-2) -- (0,2);
        \draw [line width=0.4pt,dash pattern=on 2pt off 2pt] (-2,-2) -- (2,2);
        \draw [line width=0.4pt,dash pattern=on 2pt off 2pt] (0.5,-2) -- (0.5,2);
        \draw [line width=0.4pt,dash pattern=on 2pt off 2pt] (-2,-1) -- (2,3);
        \begin{scriptsize}
            \draw [fill=black] (1,0) circle (1.2pt);
            \draw[color=black] (1.0519019508432352,0.08725485318699086) node {$\alpha_1$};
            \draw [fill=black] (-1,1) circle (1.2pt);
            \draw[color=black] (-1,1.1) node {$\alpha_2$};
            \draw [fill=uuuuuu] (0,1) circle (1.2pt);
            \draw[color=uuuuuu] (0.1,1.1) node {$\alpha_1+\alpha_2$};
            \draw [fill=uuuuuu] (1,1) circle (1.2pt);
            \draw[color=uuuuuu] (1.05,1.1) node {$2\alpha_1+\alpha_2$};
            \draw [fill=uuuuuu] (-1,0) circle (1.2pt);
            \draw[color=uuuuuu] (-1.05,0.1) node {$-\alpha_1$};
            \draw [fill=uuuuuu] (-1,-1) circle (1.2pt);
            \draw[color=uuuuuu] (1,-0.9) node {$-\alpha_2$};
            \draw [fill=uuuuuu] (0,-1) circle (1.2pt);
            \draw[color=uuuuuu] (0,-0.9) node {$-\alpha_1-\alpha_2$};
            \draw [fill=uuuuuu] (1,-1) circle (1.2pt);
            \draw[color=uuuuuu] (-1,-0.9) node {$-2\alpha_1-\alpha_2$};
            \draw [fill=ududff] (0,0) circle (1.2pt);
            \draw[color=ududff] (0.06,-0.06) node {$0$};
            \draw[color=black] (-0.35,1.4) node {$\langle\cdot,\check{\alpha_1}\rangle=0$};
            \draw[color=black] (-1,-0.4) node {$\langle\cdot,\check{\alpha_2}\rangle=1$};
            \draw [fill=ffqqqq] (0.5,0.5) circle (1.2pt);
            \draw[color=ffqqqq] (0.65,0.5) node {$\omega_1$};
            \draw [fill=ffqqqq] (0,1) circle (1.2pt);
            \draw[color=ffqqqq] (-0.33,1.085) node {$\omega_2=$};
            \draw[color=black] (0.85,1.4) node {$\langle\cdot,\check{\alpha_1}\rangle=1$};
            \draw[color=black] (-1,-1.4) node {$\langle\cdot,\check{\alpha_2}\rangle=0$};
        \end{scriptsize}
    \end{tikzpicture}
\end{center}

\vspace{0.5cm}

\begin{center}
    \begin{tikzpicture}[line cap=round,line join=round,>=triangle 45,x=2.5cm,y=2.5cm]
        \clip(-1.5,-1.5) rectangle (1.5,1.6);
        \draw [line width=0.4pt,dash pattern=on 2pt off 2pt] (0,-2) -- (0,2);
        \draw [line width=0.4pt,dash pattern=on 2pt off 2pt] (-2,-2) -- (2,2);
        \draw [line width=0.4pt,dash pattern=on 2pt off 2pt] (0.5,-2) -- (0.5,2);
        \draw [line width=0.4pt,dash pattern=on 2pt off 2pt] (-2,-1.5) -- (2,2.5);
        \draw [line width=0.4pt,dash pattern=on 2pt off 2pt] (-2,2.5) -- (2,-1.5);
        \begin{scriptsize}
            \draw [fill=black] (1,0) circle (1.2pt);
            \draw[color=black] (1.0519019508432352,0.08725485318699086) node {$\check{\alpha_1}$};
            \draw [fill=black] (-0.5,0.5) circle (1.2pt);
            \draw[color=black] (-0.55,0.6) node {$\check{\alpha_2}$};
            \draw [fill=uuuuuu] (0,1) circle (1.2pt);
            \draw[color=uuuuuu] (0,1.1) node {$\check{\alpha_1}+2\check{\alpha_2}$};
            \draw [fill=uuuuuu] (0.5,0.5) circle (1.2pt);
            \draw[color=uuuuuu] (1.14,0.5) node {$\check{\alpha_1}+\check{\alpha_2}$};
            \draw [fill=uuuuuu] (-1,0) circle (1.2pt);
            \draw[color=uuuuuu] (-1.05,0.1) node {$-\check{\alpha_1}$};
            \draw [fill=uuuuuu] (0.5,-0.5) circle (1.2pt);
            \draw[color=uuuuuu] (0.7,-0.5) node {$-\check{\alpha_2}$};
            \draw [fill=uuuuuu] (0,-1) circle (1.2pt);
            \draw[color=uuuuuu] (0,-0.9) node {$-\check{\alpha_1}-\check{\alpha_2}$};
            \draw [fill=uuuuuu] (-0.5,-0.5) circle (1.2pt);
            \draw[color=uuuuuu] (-0.55,-0.6) node {$-2\check{\alpha_1}-\check{\alpha_2}$};
            \draw [fill=ududff] (0,0) circle (1.2pt);
            \draw[color=ududff] (0.06,-0.06) node {$0$};
            \draw[color=black] (-0.35,1.4) node {$\langle\alpha_1,\cdot\rangle=0$};
            \draw[color=black] (-1.17,-0.3) node {$\langle\alpha_2,\cdot\rangle=1$};
            \draw [fill=ffqqqq] (0.5,0.5) circle (1.2pt);
            \draw[color=ffqqqq] (0.7,0.5) node {$\epsilon_1^*=$};
            \draw [fill=ffqqqq] (0,0.5) circle (1.2pt);
            \draw[color=ffqqqq] (-0.1,0.55) node {$\epsilon_2^*$};
            \draw[color=black] (0.85,1.4) node {$\langle\alpha_1,\cdot\rangle=1$};
            \draw[color=black] (-1,-1.4) node {$\langle\alpha_2,\cdot\rangle=0$};
            \draw[color=black] (-1,1) node {$\langle\alpha_0,\cdot\rangle=1$};

        \end{scriptsize}
    \end{tikzpicture}
\end{center}

\end{document}