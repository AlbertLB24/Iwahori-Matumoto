\subsection{The Lie algebra $\mathrm{sl}_2$}

Let $\fg_\CC=\mathrm{sl}_2(\CC)$ be the three-dimensional Lie algebra with basis given by
$$e=\begin{pmatrix}
    0 & 1\\
    0 & 0\\
\end{pmatrix},\quad h=\begin{pmatrix}
    1 & 0\\
    0 & -1\\
\end{pmatrix},\quad f=\begin{pmatrix}
    0 & 0\\
    1 & 0\\
\end{pmatrix},$$
and Lie bracket $[he]=2e$, $[hf]=-2f$ and $[ef]=h$. Thus, by choosing $\{e,h,f\}$ as basis of $\slii(\CC)$, the adjoint map $\mathrm{ad}:\slii(\CC)\to\mathrm{gl}(\slii(\CC))$ is given by 

$$e=\begin{pmatrix}
    0 & -2 & 0\\
    0 & 0 & 1\\
    0 & 0 & 0\\
\end{pmatrix},\quad h=\begin{pmatrix}
    2 & 0 & 0\\
    0 & 0 & 0\\
    0 & 0 & -2\\
\end{pmatrix},\quad f=\begin{pmatrix}
    0 & 0 & 0\\
    -1 & 0 & 0\\
    0 & 2 & 0\\
\end{pmatrix}.$$

We have that $\mathfrak{t}=\langle h\rangle$ is a Cartan subalgebra, so it acts on $\slii(\CC)$ by a semisimple endomorphism and we have the eigenspace decomposition 
$$\fg_\CC=\slii(\CC)=\langle h\rangle\oplus\langle e\rangle\oplus\langle f\rangle=\mathfrak{t}\oplus\fg_\alpha\oplus\fg_{-\alpha},$$
where $\alpha:\mathfrak{t}\to\CC$ satisfies $\alpha=h$. The root system of $\slii(\CC)$ is therefore $\Phi=\{\alpha,-\alpha\}$, whose root lattice and weight lattice are $Q=\ZZ\alpha$ and $P=\ZZ\frac{\alpha}{2}$ respectively.

The crucial observation is that the basis $\{e,h,f\}$ is a Chevalley basis of $\fg_\CC=\slii(\CC)$, and therefore the $\ZZ$-module $\fg_\ZZ=\ZZ h\oplus\ZZ e\oplus\ZZ f$ is a Lie algebra over $\ZZ$. Hence, by fixing a field $K$, one obtains $\fg_K:=K\otimes_\ZZ\fg_\ZZ$, naturally a Lie algebra over $K$ with Lie bracket
$$[\mu_1\otimes x_1,\mu_2\otimes x_2]:=\mu_1\mu_2\otimes[x_1,x_2],$$
making it isomorphic to $\slii(K)$.

The adjoint Chevalley group of type $A_1$ over $K$ is defined as the subgroup of $G\leq\Aut(\fg_K)$ generated by $\{x_\alpha(t),x_{-\alpha}(t),h(\chi):t\in K\text{ and }\chi\in\Hom(\ZZ\alpha,K^*)\}$ where
$$x_\alpha(t)=\exp(\mathrm{ad}(te))=\begin{pmatrix}
    1 & -2t & -t^2\\
    0 & 1 & t\\
    0 & 0 & 1\\
\end{pmatrix},\quad x_{-\alpha}(t)=\exp(\mathrm{ad}(tf))=\begin{pmatrix}
    1 & 0 & 0\\
    -t & 1 & 0\\
    -t^2 & 2t & 1\\
\end{pmatrix},\quad h(\chi)=\begin{pmatrix}
    \chi(\alpha) & 0 & 0\\
    0 & 1 & 0\\
    0 & 0 & \chi(-\alpha)\\
\end{pmatrix}.$$

Since $\chi(-\alpha)=\chi(\alpha)^{-1}$, $G$ is also a subgroup of $\SL_3(K)$. To study the structure of $G$, we denote the \textit{root subgroups} $X_\alpha=\{x_\alpha(t):t\in K\}$ and $X_{-\alpha}=\{x_{-\alpha}(t):t\in K\}$, both isomorphic to $K$, and the \textit{diagonal subgroup} (or \textit{torus}) $H=\{h(\chi):\chi\in\Hom(\ZZ\alpha,K^*)\}$, isomorphic to $K^*$. 
From the identity $h(\chi)\circ\exp(\ad(x))\circ h(\chi)^{-1}=\exp(\ad(h(\chi)(x)))$, we obtain that $H$ normalizes $X_\alpha$ and $X_{-\alpha}$. Together with the fact that $H$ is abelian, we have that $\langle X_\alpha,X_{-\alpha}\rangle=G'\triangleleft G$ is the commutator subgroup of $G$. 

We would like a more concrete description of $G$ in terms of more familiar groups. The following proposition is a first step in this direction.

\begin{proposition}
    There exists a homomorphism of groups 
    $\Psi_\alpha:\SL_2(K)\rightarrow\langle X_\alpha,X_{-\alpha}\rangle=G'$
    such that 
    $$\Psi_\alpha\left(\begin{pmatrix}
        1 & t\\
        0 & 1\\
    \end{pmatrix}\right)=x_\alpha(t)\quad\text{and}\quad\Psi_\alpha\left(\begin{pmatrix}
        1 & 0\\
        t & 1\\
    \end{pmatrix}\right)=x_{-\alpha}(t)\quad\text{for all $t\in K$}.$$
\end{proposition}
\begin{proof}
    The map $\Psi_\alpha$ can be realized by an explicit representation of $\SL_2(K)$. Consider the action of $\SL_2(K)$ on the space of polynomials $K[x,y]$ given by 
    $$\begin{pmatrix}
        a & b\\
        c & d\\
    \end{pmatrix}\cdot f(x,y)=f(ax+cy,bx+dy),$$
    and restrict this action to the three dimensional subspace $K[x,y]_2$ of degree $2$ polynomials. By choosing the basis $\{-x^2,2xy,y^2\}$, one can easily check that the action is given by the homomorphism
    $$\begin{pmatrix}
        a& b\\
        c&d\\
    \end{pmatrix}\longmapsto\begin{pmatrix}
        a^2 & -2ab & -b^2\\
        -ac & ad+bc & bd\\
        -c^2 & 2cd & d^2\\
    \end{pmatrix},$$
    and this is precisely the desired homomorphism $\Psi_\alpha$.
\end{proof}

Moreover, one can easily check that $\begin{psmallmatrix}
    a&b\\
    c&d\\
\end{psmallmatrix}\in\ker\Psi_\alpha$ if and only if $b=c=0$ and $a=d=\pm{1}$. This implies
$$G'=\langle X_\alpha,X_{-\alpha}\rangle\cong\SL_2(K)/\{\pm I\}=\PSL_2(K),$$
and the bijection is explicitly given by $\Psi_\alpha$. In addition, this gives an explicit description of the torus $H':=H\cap G'$ of $G'$ since $\Psi_\alpha\left(\begin{psmallmatrix}
    a&b\\
    c&d\\
\end{psmallmatrix}\right)$ is of the form $h(\chi)$ if and only if $b=c=0$ and $a=d^{-1}$, in which case 
$$h_\alpha(\lambda)=\Psi_\alpha\left(\begin{pmatrix}
    \lambda & 0\\
    0 & \lambda^{-1}\\
\end{pmatrix}\right)=\begin{pmatrix}
    \lambda^2 & 0 & 0\\
    0 & 1 & 0\\
    0 & 0 & \lambda^{-2}\\
\end{pmatrix}.$$

This implies that $H'=H\cap G'$ contains the elements $h(\chi)$ such that $\chi\in\Hom(\ZZ\alpha,(K^*)^2)$. Equivalently, $h(\chi)\in H'$ if and only if there is some $\bar\chi\in\Hom(\ZZ\frac{\alpha}{2},K^*)\text{ such that }\bar\chi(\alpha)=\chi(\alpha)$. This is a general phenomenon, as we shall see later.

The next natural question, of course, it whether we have a global description of $G$ similar to $G'$. This is indeed the case, as we now show.

\begin{theorem}
    There exists a unique homomorphism of groups $\overline{\Psi_\alpha}:\GL_2(K)\to\langle X_\alpha,X_{-\alpha},H\rangle=G$
    extending $\Psi_\alpha$ such that 
    $$\overline{\Psi_\alpha}\left(\begin{pmatrix}
        s & 0\\
        0 & 1\\
    \end{pmatrix}\right)=\begin{pmatrix}
        s & 0 & 0\\
        0 & 1 & 0\\
        0 & 0 & s^{-1}\\
    \end{pmatrix}\in H$$
    for all $s\in K^*$.
\end{theorem}
\begin{proof}
    We claim that the natural extension of $\Psi_\alpha$ given by 
    $$\begin{pmatrix}
        a & b\\
        c & d\\
    \end{pmatrix}\in\GL_2(K)\longmapsto\frac{1}{ad-bc}\begin{pmatrix}
        a^2 & -2ab & -b^2\\
        -ac & ad+bc & bd\\
        -c^2 & 2cd & d^2\\
    \end{pmatrix}$$
    is the desired map. A simple but tedious calculation shows that this is a group homomorphism that coincides with $\Psi_\alpha$ on $\SL_2(K)$ and maps $\begin{psmallmatrix}
        s & 0\\
        0 & 1\\
    \end{psmallmatrix}$
    to $\mathrm{Diag}(s,1,s^{-1})$. Since $$\GL_2(K)=\left\langle\begin{pmatrix}
        1 & t\\
        0 & 1\\
    \end{pmatrix},\begin{pmatrix}
        1 & 0\\
        t & 1\\
    \end{pmatrix}, \begin{pmatrix}
        s & 0\\
        0 & 1\\
    \end{pmatrix}; t\in K, s\in K^*\right\rangle,$$
    this also implies that $\Ima\overline{\Psi_\alpha}=G$, as desired.
\end{proof}

It is an easy check that $\begin{psmallmatrix}
    a & b\\
    c & d\\
\end{psmallmatrix}\in\ker\overline{\Psi_\alpha}$ if and only if $b=c=0$ and $a=d$. Hence, $$G\cong\GL_2(K)/\{\lambda I,\lambda\in K^*\}=\PGL_2(K),$$
and we write $\Psi$ for the isomorphism $G\cong \PGL_2(K)$ induced by $\overline{\Psi_\alpha}$.


Under $\Psi$, one can identify important subgroups of $G$ with subgroups of $\PGL_2(K)$. Indeed,

$$X_\alpha \longleftrightarrow\begin{pmatrix}
    1 & *\\
    0 & 1\\
\end{pmatrix},\quad X_{-\alpha} \longleftrightarrow\begin{pmatrix}
    1 & 0\\
    * & 1\\
\end{pmatrix},\quad H\longleftrightarrow\begin{pmatrix}
    * & 0\\
    0 & *\\
\end{pmatrix}.$$
Moreover, one also defines the \textit{monoidal} subgroup $N=\langle H,n_\alpha\rangle$, where $n_\alpha=\Psi_\alpha\begin{psmallmatrix}
    0 & -1\\
    1 & 0\\
\end{psmallmatrix}$ and the \textit{Borel} subgroup $B=UH$. Under $\Psi$, these correspond to
$$N\longleftrightarrow\begin{pmatrix}
    * & 0\\
    0 & *\\
\end{pmatrix}\sqcup\begin{pmatrix}
    0 & *\\
    * & 0\\
\end{pmatrix},\quad B\longleftrightarrow\begin{pmatrix}
    * & *\\
    0 & *\\
\end{pmatrix}.$$

Of course, by intersecting these subgroups with $G'$, we get the same identifications inside $\PSL_2(K)\subseteq\PGL_2(K)$. Therefore,the following results also hold if we intersect the subgroups with $G'$.

Firstly, we note that $N$ is the normalizer of $H$ inside $G$. Therefore $H\triangleleft N$ and $N/H\cong C_2$, generated by $n_\alpha H$. Finally, a simple calculation shows the following.

\begin{theorem}[Bruhat Decomposition for $\slii$]
    Let $G,B,N$ be as above. Then 
    $$G=BNB=B\sqcup Bn_\alpha B.$$    
\end{theorem}
\begin{proof}
    By using the identification given by $\Psi$, a simple calculation shows that 
    $$\begin{pmatrix}
        a_1 & b_1\\
        0 & d_1\\
    \end{pmatrix}\begin{pmatrix}
        0 & 1\\
        -1 & 0\\
    \end{pmatrix}\begin{pmatrix}
        a_2 & b_2\\
        0 & d_2\\
    \end{pmatrix}=\begin{pmatrix}
        -b_1a_2 & -b_1b_2+a_1d_2\\
        -d_1a_2 & -d_1b_2\\
    \end{pmatrix}.$$
    Since $d_1a_2\neq 0$, a simple calculation shows that $Bn_\alpha B=G\setminus B$, as desired.
\end{proof}