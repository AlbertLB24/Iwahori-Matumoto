\subsection{Root Systems}
Let $E$ be a finite dimensional vector space over $\RR$ with an inner product $(\cdot,\cdot):E\times E\to\RR$. For each $\lambda\in E$, we let $H_\lambda=\{\mu\in E:(\mu,\lambda)=0\}$ be the hyperplane perpendicular to $\lambda$ and let $$w_\lambda(\mu)=\mu-\frac{2(\mu,\lambda)}{(\lambda,\lambda)}\lambda,\quad \mu\in E$$ be the reflection along $H_\lambda$.
\begin{definition}
    A subset $\Phi\subset E$ is a \textit{root system} if
    \begin{enumerate}
        \item $\Phi$ is finite and $0\not\in\Phi$.
        \item $\Phi$ spans $E$.
        \item For all $\alpha,\beta\in\Phi$, $\frac{2(\alpha,\beta)}{(\alpha,\alpha)}\in\ZZ$.
        \item For all $\alpha\in\Phi$, $w_\alpha(\Phi)=\Phi$.
        \item If $\alpha\in\Phi$, then $\Phi\cap\ZZ\alpha=\{\alpha,-\alpha\}$.
    \end{enumerate}
\end{definition} 

It is clear from the definition that the reflections $w_\alpha$ induce automorphisms of the root system. 

\begin{definition}
    The \textit{Weyl group} $W_\Phi$ of the root system $\Phi$ is defined as the group generated by $\{w_\alpha:\alpha\in\Phi\}$.
\end{definition}

Since $E$ comes equipped with an inner product, there is a canonical isomorphism $E\cong E^*$ given by $\lambda\mapsto\lambda^*$ where $\lambda^*(\mu)=(\lambda,\mu)$. Furthermore, the inner product naturally extends to $E^*$ by declaring that the isomorphism above is an isometry; that is, by defining $(\lambda^*,\mu^*)=(\lambda,\mu)$. 

Furthermore, the isomorphism $E\rightarrow E^{**}$ given by $\lambda\mapsto\lambda^{**}$ coincides with the evaluation map $\mathrm{ev}:E\rightarrow E^{**}$ satisfying $\mathrm{ev}(\lambda)(f)=f(\lambda)$ for all $\lambda\in E$ and $f\in E^*$. Indeed, if $\mu\in E$ satisfies that $\mu^*=f$, then 
$$\lambda^{**}(f)=(\lambda^*,f)=(\lambda^*,\mu^*)=(\lambda,\mu)=\mu^*(\lambda)=f(\lambda)=\mathrm{ev}(\lambda)(f).$$

Given $\alpha\in\Phi$, we can define $\check{\alpha}=\frac{2\alpha^*}{(\alpha,\alpha)}\in E^*$. Then conditions $3.$ and $4.$ above can be rephrased as
\begin{enumerate}[start=3]
    \item For all $\alpha,\beta\in\Phi$, $\calpha(\beta)\in\ZZ$.
    \item For all $\alpha,\beta\in\Phi$, $w_\alpha(\beta)=\beta-\calpha(\beta)\beta\in\Phi$.
\end{enumerate}

This notation is useful due to the following fact:

\begin{lemma}
    Let $\cPhi=\{\calpha:\alpha\in\Phi\}\subset E^*$. Then $\cPhi$ is a root system of $E^*$, denoted the \textit{dual root system}.
\end{lemma}
\begin{proof}
    Clearly, $\cPhi$ is a finite subset, and if it does not span $E^*$, then there is some $\mu\in E\setminus\{0\}$ such that $\calpha(\mu)=0$ for all $\alpha\in\Phi$. This means that $(\mu,\alpha)=0$ for all $\alpha\in\Phi$, so $\mu$ lies in the orthogonal complement of $\mathrm{Span}_\RR(\Phi)=E$, a contradiction.

    Next, we note that if $\alpha,\beta\in\Phi$, then 
    $$(\calpha,\cbeta)=\frac{4(\alpha,\beta)}{(\alpha,\alpha)(\beta,\beta)},$$
    and therefore 
    $$\check{\calpha}(\cbeta)=\frac{2(\calpha,\cbeta)}{(\calpha,\calpha)}=\frac{2(\alpha,\beta)}{(\beta,\beta)}=\cbeta(\alpha)\in \ZZ.$$
    This calculation also shows that 
    $$w_{\calpha}(\cbeta)=\cbeta-\check{\calpha}(\cbeta)\calpha=\frac{2\beta^*}{(\beta,\beta)}-\frac{2(\alpha,\beta)}{(\beta,\beta)}\frac{2\alpha^*}{(\alpha,\alpha)}=\frac{2(\beta^*-\calpha(\beta)\alpha^*)}{(w_\alpha(\beta),w_\alpha(\beta))}=\frac{2w_\alpha(\beta)^*}{(w_\alpha(\beta),w_\alpha(\beta))}=\check{w_\alpha(\beta)}\in\cPhi.$$
    Finally, if $\calpha\in\cPhi$ and $c\calpha=\check{(\alpha/c)}\in\cPhi$ for some $c\in\RR$, then $\alpha/c\in\Phi$ so $c=\pm1$.
\end{proof}

\begin{proposition}
    The function
    \begin{align*}
        \phi:W_{\Phi}&\longrightarrow W_{\cPhi}\\
        w_\alpha&\longmapsto w_{\calpha}
    \end{align*}
    extended multiplicatively induces a well-defined isomorphism of groups.
\end{proposition}
\begin{proof}
    We claim that for any $w\in W_\Phi$, $\phi(w)=\check{w}$, where $\check{w}(\cbeta)=\check{w(\beta)}$ for any $\beta\in\Phi$. This is indeed the case for $w=w_\alpha$ for any $\alpha$ since $w_{\calpha}(\cbeta)=\check{w_\alpha(\beta)}$ for any $\beta\in\Phi$. 
    
    Suppose now that $\phi(w)=\check{w}$ for some $w\in W_\Phi$ and let $\alpha\in\Phi$. Then $$\phi(ww_\alpha)(\cbeta)=\phi(w)w_{\calpha}(\cbeta)=\check{w}(\check{w_\alpha(\beta)})=\check{ww_\alpha(\beta)}.$$
    Since $W_\Phi$ is generated by $\{w_\alpha:\alpha\in\Phi\}$, a simple inductive argument proves the claim. This description of $\phi(w)=\check{w}$ is generator-free, and hence $\phi$ is a well-defined group homomorphism. It is surjective since $W_{\cPhi}$ is generated by $\{w_{\calpha}:\calpha\in\cPhi\}$ and is injective since $\check{w}(\cbeta)=\cbeta$ for all $\beta\in\Phi$ if and only if $w(\beta)=\beta$ for all $\beta\in\Phi$.
\end{proof}

From now on, we will simply write $W$ whenever the underlying root system is clear from context.

\subsection{Properties of the Weyl Group}

Fix some root system $\Phi\subset E$, and for each $\alpha\in\Phi$, let $H_\alpha=\{\lambda\in E:(\alpha,\lambda)=0\}$ be the hyperplane perpendicular to $\alpha$. An element $\gamma\in E\setminus\cup_{\alpha\in\Phi}H_\alpha$ is said to be \textit{regular} and connected components of $E\setminus\cup_{\alpha\in\Phi}H_\alpha$ are called \textit{Weyl chambers}. 

Finally, we say that a subset $\Delta=\{\alpha_1,\ldots,\alpha_l\}\subset\Phi$ is an \textit{integral basis} if $\Delta$ is a basis of $E$ and any $\alpha=\sum_{i=1}^l c_i\alpha_i\in\Phi$ satisfies that either all $c_i\in\ZZ^{\geq 0}$ or all $c_i\in\ZZ^{\leq 0}$. If such a subset $\Delta$ exists, then we distinguish between \textit{positive roots} $\Phi^+$ (if all $c_i\in\ZZ^{\geq0}$) and \textit{negative roots} $\Phi^-$ (if all $c_i\in\ZZ^{\leq0}$).

The aim of this subsection is to prove the following theorem that describes the structure of the Weyl group.

\begin{theorem}\label{thm:weyl}
    Let $\Phi\subset E$ be a root system and let $W$ be its Weyl group. Then
    \begin{enumerate}[label=(\arabic*)]
        \item $\Phi$ contains an integral basis $\Delta=\{\alpha_1,\ldots,\alpha_l\}$.
        \item There is a 1-to-1 correspondence between integral basis and Weyl chambers.
        \item $W$ is generated by $\{w_{\alpha_1},\ldots,w_{\alpha_l}\}$.
        \item $W$ acts simply transitively on Weyl chambers (and also on integral basis).
    \end{enumerate}
\end{theorem}

\begin{proof}[Proof of Theorem \ref{thm:weyl}(1), (2) and (3)]
    Let $\gamma$ be some regular element and let $\Phi^+(\gamma)=\{\alpha\in\Phi:(\gamma,\alpha)>0\}$. Define $$\Delta(\gamma):=\{\alpha\in\Phi^+(\gamma):\text{ there are no }\beta_1,\beta_2\in\Phi^+(\gamma) \text{ such that }\alpha=\beta_1+\beta_2\}.$$ 
    Now suppose that not all $\alpha\in\Phi^+(\gamma)$ is a $\ZZ^{\geq0}$ linear combination of $\Delta(\gamma)$ and choose such a $\beta$ that minimizes $(\beta,\gamma)$. Since $\beta\not\in\Delta(\gamma)$, there are $\beta_1,\beta_2\in\Phi^+(\gamma)$ such that $\beta=\beta_1+\beta_2$. But $(\beta_i,\gamma)<(\beta,\gamma)$ for $i=1,2$ so $\beta_1,\beta_2$ are a $\ZZ^{\geq0}$ linear combination of $\Delta(\gamma)$, contradicting the choice of $\beta$. Hence, all $\alpha\in\Phi^+(\gamma)$ are a $\ZZ^{\geq0}$ linear combination of $\Delta(\gamma)$. Since $\Phi=\Phi^+(\gamma)\cup-\Phi^+(\gamma)$, all $\alpha=\sum_{i=1}^l c_i\alpha_i\in\Phi$ satisfies that either all $c_i\in\ZZ^{\geq 0}$ or all $c_i\in\ZZ^{\leq 0}$.

    Next, we need to show that $\Delta(\gamma)$ is a basis. Let $\alpha,\beta\in\Delta(\gamma)$ and suppose $(\alpha,\beta)>0$. Then $\pm(\alpha-\beta)\in\Phi$ so either $\alpha-\beta\in\Phi^+(\gamma)$ or $\beta-\alpha\in\Phi^+(\gamma)$. In the former case, $\alpha=(\alpha-\beta)+\beta$ and in the latter, $\beta=(\beta-\alpha)+\alpha$. Either case contradicts the definition of $\Delta(\gamma)$, so $(\alpha,\beta)\leq0$.
    Suppose that $$v=\sum_{i=1}^rc_i\alpha_i=\sum_{j=1}^sd_j\beta_j\in E$$ where $\Delta(\gamma)=\{\alpha_1,\ldots,\alpha_r,\beta_1,\ldots,\beta_s\}$ and all $c_i,d_j\geq0$. Then
    $$0\leq(v,v)=\sum_{i=1}^r\sum_{j=1}^sc_id_j(\alpha_i,\beta_j)\leq0,$$
    so $v=0$. Finally, by definition of $\Phi^+(\gamma)$,
    $$0=(\gamma,v)=\sum_{i=1}^{r}c_i(\gamma,\alpha_i)=\sum_{j=1}^{s}d_j(\gamma,\beta_j)$$
    implies that $c_i=d_j=0$ for all $1\leq i\leq r$ and $1\leq j\leq s$. This concludes the proof of (1).

    It is clear the definition of $\Phi^+(\gamma)$ only depends on the Weyl chamber $\gamma$ lies in, so the above proof gives a way to construct an integral basis from a Weyl chamber. Conversely, given an integral basis $\Delta=\{\alpha_1,\ldots,\alpha_l\}$ the intersection of half spaces $\cap_{i=1}^l\{\lambda\in E:(\lambda,\alpha_i)>0\}$ is a Weyl chamber (in particular, it is non-empty), denoted as $C(\Delta)$. It is clear that $\gamma\in C(\Delta(\gamma))$ and that if $\gamma\in C(\Delta)$, then $\Delta=\Delta(\gamma)$, so the constructions are inverses of each other. This concludes the proof of (2).

    We now fix an integral basis $\Delta=\{\alpha_1,\ldots,\alpha_l\}$ of $\Phi$ and we define $W_0$ to be the subgroup of $W$ generated by $\{w_{\alpha_1},\ldots,w_{\alpha_l}\}$. To prove (3), we first observe that $W$ acts on Weyl chambers, and we prove that $W_0$ acts transitively on them. A regular element $\mu$ lies in $C(\Delta)$ if and only if $(\mu,\alpha_i)$ for all $1\leq i\leq l$ and since $\Delta$ is a basis of $E$, it follows that the faces of $C(\Delta)$ are the hyperplanes $H_{\alpha_1},\ldots,H_{\alpha_l}$. Let $C$ is any Weyl chamber and let $\lambda\in C$, $\mu\in C(\Delta)$. Since the orbit of $\lambda$ under the action of $W_0$ is discrete, we may choose some $w\in W_0$ such that 
    $$|w(\lambda)-\mu|=\inf_{x\in W_0\lambda}|x-\mu|.$$
    If $w(\lambda)\not\in C(\Delta)$, then there is some $j\in\{1,\ldots,l\}$ such that $H_{\alpha_j}$ lies between $w(\lambda)$ and $\mu$. But $H_{\alpha_j}$ is the perpendicular bisector of $w(\lambda)$ and $w_{\alpha_j}w(\lambda)$, so 
    $$|w_{\alpha_j}w(\lambda)-\mu|<|w(\lambda)-\mu|,$$
    which is a contradiction. Thus, $w(\lambda)\in C(\Delta)$ and $w(C)\cap C(\Delta)\neq\emptyset$ so $w(C)=C(\Delta)$.

    Furthermore, since $w(\Delta)$ is an integral basis for any $w\in W$ and $w(C(\Delta))=C(w(\Delta))$, $W$ also acts transitively on the set of integral basis.

    Finally, we show that $W=W_0$. Fix some $\alpha\in\Phi$ and suppose there is some $w\in W_0$ such that $w(\alpha)\in\Delta$. Then $w_\alpha=w^{-1}w_{w(\alpha)}w\in W_0$. Thus it suffices to show that such $w\in W_0$ exists for all $\alpha\in\Phi$. To prove this, consider some $\gamma'\in H_\alpha$ but not lying in any $H_\beta$ for any $\beta\neq\pm\alpha$. Then choose some $\gamma$ sufficiently chose to $\gamma'$ so that $0<(\gamma,\alpha)<|(\gamma,\beta)|$ for any $\beta\neq\pm\alpha$. Then it is clear that $\alpha\in\Delta(\gamma)$ and since $W_0$ acts transitively on the integral basis, there exists some $w\in W_0$ such that $w(\Delta(\gamma))=\Delta$. In particular, $w(\alpha)\in\Delta$ as wished.
\end{proof}

It only remains to show that the action is simple; that is, if $w\in W$ satisfies $w(C(\Delta))=C(\Delta)$, then $w=1$. First, we need to introduce some vocabulary. The roots $\alpha_i\in\Delta$ are called \textit{simple roots} and the reflections $w_{\alpha_i}\in W$ are called \textit{simple reflections}. The \textit{height} of a root $\alpha=\sum_{i=1}^{l}c_i\alpha_i$ is defined as $\hht(\alpha)=\sum_{i=1}^{l}c_i$.

To simply notation, we denote $w_i$ for $w_{\alpha_i}$, and we remark that the set $\{w_1,\ldots,w_l\}$ is a \textit{minimal} set of generators of $W$. For each $w\in W$, we let $n(w)=\{\alpha\in\Phi:w(\alpha)\in\Phi^-\}=|\Phi^+\cap w^{-1}(\Phi^-)|$
and
$$l(w)=\min\{r\in\ZZ^{\geq0}:w=w_{i_1}\ldots w_{i_r} \text{ and all } w_{i_j} \text{ are simple reflections}\},$$
called the \textit{length} of the element $w$.

\begin{proposition}
    For all $w\in W$, we have $n(w)=l(w)$.
\end{proposition}

\begin{proof}[Proof of Theorem \ref{thm:weyl}(4)]
    Suppose that $w(C(\Delta))=C(\Delta)$. Then $w(\Delta)=\Delta$ and therefore $w(\Phi^+)=\Phi^+$. This implies that $l(w)=n(w)=0$ and therefore $w=1$ as required. This concludes the proof of Theorem \ref{thm:weyl}.
\end{proof}





\subsection{The Weight Lattice}

We begin this subsection with a basic result about simple reflections.

\begin{lemma}\label{lem:simplereflection}
    Let $\alpha_k\in\Delta\subset\Phi$ be a simple a root. Then $w_{\alpha_k}(\Phi^+\setminus\{\alpha_k\})=\Phi^+\setminus\{\alpha_k\}$.
\end{lemma}

\begin{proof}
    Let $\beta$ be a positive root other than $\alpha_k$. Then $\beta=\sum_{i=1}^{l}c_i\alpha_i$ and $c_j> 0$ for some $j\neq k$. Then the coefficient of $\alpha_j$ in $w_{\alpha_k}(\beta)=\beta-\check{\alpha_k}(\beta)\alpha_i\in\Phi$ is also $c_j>0$. Hence, $w_{\alpha_k}(\beta)\in\Phi^+\setminus\{\alpha_k\}$ since $\beta\neq-\alpha_k$.
\end{proof}

We now introduce the fundamental notion of a weight.

\begin{definition}
    Let $\Phi\subset E$ be a root system and let $\Delta=\{\alpha_1,\ldots,\alpha_l\}$ be an integral basis. Then the \textit{root lattice} $Q$ is $\ZZ\Phi=\oplus_{i=1}^l\ZZ\alpha_i$ and the \textit{weight lattice} is 
    $$P=\{\lambda\in E:\calpha(\lambda)\in\ZZ\text{ for all }\alpha\in\Phi\}.$$
    The elements of the weight lattice are called \textit{weights}.
\end{definition}

\begin{lemma}
    Let $\Delta=\{\alpha_1,\ldots,\alpha_l\}$ be an integral basis of a root system $\Phi\subset E$. Then $\check{\Delta}=\{\check{\alpha_1},\ldots,\check{\alpha_l}\}$ is an integral basis of $\cPhi\subset E^*$. Furthermore,
    $$P=\{\lambda\in E:\check{\alpha_i}(\lambda)\in\ZZ\text{ for all }1\leq i\leq l\}$$
\end{lemma}
\begin{proof}
    %Let $\alpha\in\Phi$ and assume first that $\alpha\in\Phi^+$, so $\alpha=\sum_{i=1}^{l}c_i\alpha_i$ for non-negative integers $c_1,\ldots,c_l$. Then 
    %$$\calpha=\frac{2\alpha^*}{(\alpha,\alpha)}=\sum_{i=1}^{l}\frac{2c_i}{(\alpha,\alpha)}\alpha_i^*=\sum_{i=1}^{l}\frac{(\alpha_i,\alpha_i)c_i}{(\alpha,\alpha)}\check{\alpha_i},$$
    %which implies that $\alpha$ is a $\QQ^{\geq0}$ linear combination of $\check{\Delta}$. 
    Let $\alpha\in\Phi^+$. We prove the result by induction over $\hht(\alpha)$. If $\hht(\alpha)=1$, then $\alpha$ is a simple root and the result is obvious. If $\hht(\alpha)\geq2$, we note that there is some $1\leq j\leq l$ such that $(\alpha,\alpha_j)>0$ as otherwise $\alpha\in -C(\Delta)$, a contradiction. Hence, $\check{\alpha}(\alpha_j)\in\ZZ^{>0}$, $w_{\alpha_j}(\alpha)\in\Phi^+$ and $\hht(w_{\alpha_j}(\alpha))<\hht(\alpha)$. By induction, $$\check{w_{\alpha_j}(\alpha)}=\sum_{i=1}^{l}c_i\check{\alpha_i}\text{ for some }c_i\in\ZZ^{\geq0}.$$
    On the other hand, $\check{w_{\alpha_j}(\alpha)}=w_{\check{\alpha_j}}(\calpha)=\calpha-\check{\check{\alpha_j}}(\calpha)\check{\alpha_j}=\calpha-\calpha(\alpha_j)\check{\alpha_j},$ so 
    $$\calpha=\calpha(\alpha_i)\check{\alpha_j}+\sum_{i=1}^{l}c_i\check{\alpha_i},$$
    proving the result. If $\alpha\in\Phi^-$, the argument is identical with $c_i\in\Phi^-$.

    Consequently, if $\lambda\in E$ satisfies that $\check{\alpha_i}(\lambda)\in\ZZ$ for all $1\leq i\leq l$, then $\calpha(\lambda)\in\ZZ$ for all $\alpha\in\Phi$. Thus, 
    $$P=\{\lambda\in E:\check{\alpha_i}(\lambda)\in\ZZ\text{ for all }1\leq i\leq l\},$$
    and the proof is complete
\end{proof}


\begin{definition}
    Let $\Phi$ be a root system with integral basis $\Delta=\{\alpha_1,\ldots,\alpha_l\}$. The elements $\{\omega_1,\ldots,\omega_l\}\subset E$ such that $\check{\alpha_i}(\omega_j)=\delta_{ij}$ for all $1\leq i,j\leq l$ are called the \textit{fundamental weights} of $\Phi$.
\end{definition}

It is clear from the definitions that $P=\oplus_{i=1}^l\ZZ\omega_i$. The fundamental weights satisfy the following identity.

\begin{lemma}
    Let $\delta=\frac{1}{2}\sum_{\alpha\in\Phi^+}\alpha$. Then for all simple roots $\alpha_k\in\Delta$, $w_{\alpha_k}(\delta)=\delta-\alpha_k$ and $\check{\alpha_k}(\delta)=1$. In particular, $\delta=\sum_{i=1}^{l}\omega_i$.
\end{lemma}
\begin{proof}
    This is a simple calculation using Lemma \ref{lem:simplereflection}. We note
    $$w_{\alpha_k}(\delta)=\frac{1}{2}+\sum_{\alpha\in\Phi^+\setminus\{\alpha_k\}}\frac{1}{2}w_{\alpha_k}(\alpha)=-\frac{1}{2}\alpha_k+\sum_{\alpha\in\Phi^+\setminus\{\alpha_k\}}\frac{1}{2}\alpha=\delta-\alpha_k,$$
    and this immediately implies that $\check{\alpha_k}(\delta)=1$. Hence, $\delta\in P$ and $\delta=\sum_{i=1}^{l}\omega_i$ as desired.
\end{proof}




