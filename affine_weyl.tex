Throughout this section, we fix a root system $\Phi$ on an Euclidean space $E$ with integral basis $\Delta=\{\alpha_1,\ldots,\alpha_l\}$ and Weyl group $W$. In Chapter \ref{chp:weylgroup}, we introduced two full rank lattices of $E$; the root lattice $Q=\oplus_{i=1}^l\ZZ\alpha_i$ and the weight lattice $P=\oplus_{i=1}^l\ZZ\omega_i$, which contains the root lattice $Q$. We now construct the dual lattices in $E^*$.

\begin{definition}
    Let $\Phi\subset E$ be a root system with root lattice $Q$ and weight lattice $P$. Then the \textit{dual root lattice} is 
    $$Q^\perp=\{x\in E^*:\langle\lambda,x\rangle\in\ZZ\text{ for all }\lambda\in Q\}\subset E^*$$
    and the \textit{dual weight lattice} is 
    $$P^\perp=\{x\in E^*:\langle\lambda,x\rangle\in\ZZ\text{ for all }\lambda\in P\}\subset E^*.$$
\end{definition}

The following lemma shows that the dual lattices have desirable properties with respect to the dual root system $\cPhi$. Recall that $\check{\Delta}=\{\check{\alpha_1},\ldots,\check{\alpha_l}\}$ is an integral basis of $\cPhi$.

\begin{lemma}
    The dual weight lattice is the root lattice of $\cPhi$ and the dual root lattice is the weight lattice of $\cPhi$. That is,
    $$P^\perp=\bigoplus_{i=1}^l\ZZ\check{\alpha_i}\quad\text{and}\quad Q^\perp=\bigoplus_{i=1^l}\ZZ\epsilon_i^*,$$
    where $\langle\epsilon_j^*,\check{\check{\alpha_i}}\rangle=\delta_{ij}$ for all $1\leq i,j\leq l$.
\end{lemma}
\begin{proof}
    The first statement follows directly from the fact that $P=\oplus_{i=1}^l\ZZ\omega_i$ and that $\langle\omega_j,\check{\alpha_j}\rangle=\delta_{ij}$. The second statement follows from the fact that $Q=\oplus_{i=1}^l\ZZ\alpha_i$ and $\langle\alpha_i,\epsilon_j^*\rangle=\langle\epsilon_j^*,\check{\check{\alpha_i}}\rangle=\delta_{ij}$.
\end{proof}

For each $\alpha\in\Phi$ and $k\in\ZZ$, define the hyperplane
$$H_{\alpha,k}=\{x\in E^*:\langle\alpha,x\rangle=k\}\subset E^*$$
and we consider the reflections along each $H_{\alpha,k}$ given by
$$w_{\alpha,k}(x)=x-\langle\alpha,x\rangle\calpha+k\calpha,\quad x\in E^*.$$
\begin{definition}
    The \textit{affine Weyl group} of $\Phi$, denoted by $W_a$, is the group generated by all $\{w_{\alpha,k}:\alpha\in\Phi, k\in\ZZ\}$.
\end{definition}

Since, $w_{\alpha,0}=w_{\check{\alpha}}$, it follows that $W_\Phi\cong W_{\cPhi}$ is naturally a subgroup of $W_a$. Given $d\in Q^\perp$, we consider the translation map $T(d):x\mapsto x+d,\ d\in E^*$ and we define the abelian groups 
$$D=\{T(d):d\in Q^\perp\}\quad\text{and}\quad D'=\{T(d):d\in P^\perp\}.$$
It is immediate by the definitions that $$w_{\alpha,k}=T(k\calpha)\circ w_{\alpha,0}\quad\text{and}\quad w\circ T(d)\circ w^{-1}=T(w(d))\quad \text{for all}\quad\alpha\in\Phi, k\in\ZZ, d\in Q^\perp\text{ and }w\in W.$$
Moreover, $D\cap W=D'\cap W={1}$ and hence $W_a=D'\rtimes W$.

\begin{definition}
    The \textit{extended affine Weyl group} is the group $\eW:=D\rtimes W$.
\end{definition}

We remark that $W_a=D'W$ is a normal subgroup of $\eW$. Indeed, if $T(d)\in D$, then $$T(d)w_{\alpha,k}T(-d)=T(d+k\calpha) w_{\alpha,0}T(-d)=T(d+k\calpha-w_{\alpha,0}(d))w_{\alpha,0}=T((k+\langle\alpha,d\rangle)\calpha)w_{\alpha,0}=w_{\alpha,k+\langle\alpha,d\rangle}\in W_a$$
since $d\in Q^\perp$ so $\langle\alpha,d\rangle\in\ZZ$.

\iffalse
\begin{lemma}\label{lem:normalweyl}
    The affine Weyl group $W_a$ is a normal subgroup of the extended affine Weyl group $\eW$. Moreover,
    $$\eW/W\cong D/D'\cong Q^\perp/P^\perp\cong P/Q$$
    is the fundamental group associated to $\Phi$. 
\end{lemma}

\begin{proof}
    Firstly, we note that given any $\sigma=T(d)w\in DW$, we can recover $d\in Q^\perp$ and $w\in W$ by $d=\sigma(0)$ and $w=\sigma\circ(T(-d))$. Then, let $T(d)w\in DW$ and $T(d')w'\in D'W$ for $d\in Q^\perp$ and $d'\in P^\perp$. By the initial observation, $T(d)wT(d')(T(d)w)^{-1}\in D'W$ if and only if
    $$T(d)wT(d')(T(d)w)^{-1}(0)=ww'w^{-1}(-d)+w(d')+d\in P^\perp.$$
    Since $d'\in P^\perp$ and $W$ acts on $P^\perp$, the result follows immediately from the following lemma:
    \begin{lemma}
        Let $d\in Q^\perp$ and $w\in W$. Then $d-w(d)\in P^\perp$.
    \end{lemma}
    \begin{proof}
        Since $Q=\oplus_{i=1}^l\ZZ\epsilon_i^*$ and $w\in W$ is linear, it is enough to prove the result for $d=\epsilon_k^*,1\leq k\leq l$. Furthermore, since $W$ is generated by $\{w_{1},\ldots,w_{l}\}$, where $w_i=w_{\check{\alpha_i}}$, we may write $w=w_{i_1}\cdots w_{i_l}$ as a product of simple reflections. We can express $\epsilon_k^*-w(\epsilon_k^*)$ as a telescopic sum
        $$\epsilon_k^*-w(\epsilon_k^*)=\sum_{j=1}^r w^{(j)}(\epsilon_k^*-w_{i_j}\epsilon_k^*)\text{ where } w^{(j)}=w_{i_1}\cdots w_{i_{j-1}}.$$
        By definition, we have that $\epsilon_k^*-w_{i_j}(\epsilon_k^*)=\langle\epsilon_k^*,\check{\check{\alpha_{i_j}}}\rangle\alpha_{i_j}\in P^\perp$, and since $W$ acts on  $P^\perp$ the result follows.        
    \end{proof}
    This concludes the proof of Lemma \ref{lem:normalweyl}
\end{proof}
\fi

Analogously to the standard Weyl group, we consider the connected components of $E^*\setminus\bigcup_{\substack{\alpha\in\Phi\\k\in\ZZ}}H_{\alpha,k}$, called \textit{alcoves}. We remark that $\eW=D\rtimes W$ acts on the set of alcoves. Furthermore, the set $A(\Delta)=\{x\in E:\langle\alpha,x\rangle\in(0,1)\text{ for all }\alpha\in\Phi^+\}\subset C(\Delta)$ is called the \textit{fundamental alcove} and it can also be described as 
$$A(\Delta)=\{x\in E:\langle\alpha_i,x\rangle>0\text{ for all }1\leq i\leq l\text{ and }\langle\alpha_0,x\rangle<1\}.$$
Hence the faces of the fundamental alcove $A(\Delta)$ (called \textit{facets}) are $H_{\alpha_0,1},H_{\alpha_1,0},\ldots,H_{\alpha_l,0}$. To simplify notation, we let $H_0=H_{\alpha_0,1}$ and $H_i=H_{\alpha_i,0}$ for $1\leq i\leq l$. Similarly, we let $w_i=w_{\alpha_i,0}$ and $w_0=w_{\alpha_0,1}=T(\check{\alpha_0})\circ w_{\alpha_0,0}$.

The following structure theorem is the analogous version of Theorem \ref{thm:weyl} for the affine Weyl group.

\begin{theorem}\label{thm:affineweyl}
    The affine Weyl group $W_a=D'\rtimes W$ is generated by $\{w_0,w_1,\ldots,w_l\}$ and this is a minimal set of generators. Moreover, $W_a$ acts simply transitively on the set of alcoves.
\end{theorem}
Similarly to the proof of Theorem \ref{thm:weyl}, the hardest fact to prove is that the action on the alcoves is simple.

\begin{proof}
    We proceed in a similar fashion to Theorem \ref{thm:weyl}. Consider the subgroup $W_{a,0}$ of $W_a$ generated by $\{w_0,w_1,\ldots,w_l\}$ and fix an alcove $A$. Choose some $x\in A$ and $y\in A(\Delta)$. Since the orbit of $x$ under $W_{a,0}$ is discrete, we may choose some $w\in W_{a,0}$ so that 
    $|w(x)-y|=\inf_{z\in W_{a,0}x}|z-y|$. If $w(x)\not\in A(\Delta)$, then there is some hyperplane $H_i, 0\leq i\leq l$ between $w(x)$ and $y$. But $H_i$ is the perpendicular bisector of $w_iw(x)$ and $w(x)$, and this implies that
    $$|w_iw(x)-y|<|w(x)-y|,$$
    a contradiction since $w_iw\in W_{a,0}$. Thus, $w(A)\cap A(\Delta)\neq\emptyset$, so $w(A)=A(\Delta)$ and this shows that the action of $W_{a,0}$ on the set of alcoves is transitive.

    Now fix some $\alpha\in\Phi$ and $k\in\ZZ$ and consider an alcove $A$ such that $H_{\alpha,k}$ is a facet of $A$. Choose some $w\in W_{a,0}$ such that $w(A)=A(\Delta)$. Then $w(H_{\alpha_k})=H_i$ for some $0\leq i\leq l$ and therefore $w_{\alpha,k}=w^{-1}w_iw\in W_{a,0}$. This proves that $W_a=W_{a,0}$ is generated by $\{w_0,w_1,\ldots,w_l\}$ as desired. To prove that the action is simple, we need a few preparations.
\end{proof}

Analogously to the Weyl group, given $w\in W_a$, we define the \textit{length} of $w$ to be 
$$l(w)=\min\{r\in\ZZ^{\geq0}:w=w_{i_1}\cdots w_{i_r}\text{ and all }w_{i_j}\in\{w_0,w_1,\ldots,w_l\}\}.$$
In this setting, we also need a generalization of the function $n(w)=|\cPhi^+\cap w^{-1}(\cPhi^-)|=|\cPhi^+\cap w(\cPhi^-)|,\ w\in W$ for elements of $W_a$. To that aim given a hyperplane $H_{\alpha,k}$ and alcoves $A_1$ and $A_2$, we say that $A_1\sim A_2\ (H_{\alpha,k})$ if $A_1$ and $A_2$ are in the same side of $H_{\alpha,k}$ and $A_1\nsim A_2\ (H_{\alpha,k})$ otherwise. Given $\sigma\in \eW$, we define 
$$\mathcal{H}(\sigma)=\{H_{\alpha,k}:\alpha\in\Phi,k\in\ZZ\text{ and }A(\Delta)\nsim \sigma(A(\Delta))\ (H_{\alpha,k})\}.$$
We note that this is a finite set, so we define $n(\sigma)=|\mathcal{H}(\sigma)|$. This indeed generalizes $n(w)$ for $w\in W$ since for $\calpha\in\cPhi^+$, we have that $H_{\alpha,k}\in\mathcal{H}(w)$ if and only if $k=0$ and $w^{-1}(\calpha)\in\cPhi^-$. The proof that the action of $W_a$ is simple is a direct consequence of the following proposition.

\begin{proposition}
    Let $\sigma\in W_a$ with $l(\sigma)=r$ and suppose that $\sigma=w_{i_1}\cdots w_{i_r}$ is a reduced expression. Then
    $$\mathcal{H}(\sigma)=\{H_{i_1},w_{i_1}(H_{i_2}),w_{i_1}w_{i_2}(H_{i_3}),\ldots,w_{i_1}\cdots w_{i_{r-1}}(H_{i_r})\},$$
    and all these hyperplanes are distinct. In particular, $n(\sigma)=r=l(w)$.
\end{proposition}

We are now ready to finish the proof of Theorem \ref{thm:affineweyl}

\begin{proof}[Proof of Theorem \ref{thm:affineweyl}]
    We already know that the action of $W_a$ on the set of alcoves is transitive. If $\sigma\in W_a$ satisfies that $\sigma(A(\Delta))=A(\Delta)$, then $\mathcal{H}(\sigma)=\emptyset$ and $l(\sigma)=n(\sigma)=|\mathcal{H}(\sigma)|=0$. Therefore, $\sigma=1$ and the proof is complete.
\end{proof}
































\newpage
\subsection{Examples}
\vspace{0.5cm}
\definecolor{ffqqqq}{rgb}{1,0,0}
\definecolor{ududff}{rgb}{0.30196078431372547,0.30196078431372547,1}
\definecolor{uuuuuu}{rgb}{0.26666666666666666,0.26666666666666666,0.26666666666666666}
\begin{center}
    \begin{tikzpicture}[line cap=round,line join=round,>=triangle 45,x=2.5cm,y=2.5cm]
        \clip(-1.3,-1.3) rectangle (1.3,1.5);
        \draw [line width=0.4pt,dash pattern=on 2pt off 2pt] (0,-1.7678450876512102) -- (0,2.2749034653178044);
        \draw [line width=0.4pt,dash pattern=on 2pt off 2pt,domain=-2.599612871193306:3.164564097878663] plot(\x,{(-0--0.5*\x)/0.8660254037844386});
        \draw [line width=0.4pt,dash pattern=on 2pt off 2pt] (0.5,-1.7678450876512102) -- (0.5,2.2749034653178044);
        \draw [line width=0.4pt,dash pattern=on 2pt off 2pt,domain=-2.599612871193306:3.164564097878663] plot(\x,{(--0.28867513459481287--0.28867513459481287*\x)/0.5});
        \begin{scriptsize}
            \draw [fill=black] (1,0) circle (1.2pt);
            \draw[color=black] (1.0519019508432352,0.08725485318699086) node {$\alpha_1$};
            \draw [fill=black] (-0.5,0.8660254037844386) circle (1.2pt);
            \draw[color=black] (-0.4478273510646299,0.9544896234206665) node {$\alpha_2$};
            \draw [fill=uuuuuu] (0.5,0.8660254037844386) circle (1.2pt);
            \draw[color=uuuuuu] (0.8,0.8) node {$\alpha_1+\alpha_2$};
            \draw [fill=uuuuuu] (-1,0) circle (1.2pt);
            \draw[color=uuuuuu] (-1.05,0.1) node {$-\alpha_1$};
            \draw [fill=uuuuuu] (0.5,-0.8660254037844386) circle (1.2pt);
            \draw[color=uuuuuu] (0.5498186628132109,-0.7799799170466849) node {$-\alpha_2$};
            \draw [fill=uuuuuu] (-0.5,-0.8660254037844386) circle (1.2pt);
            \draw[color=uuuuuu] (-0.5,-0.7799799170466849) node {$-\alpha_1-\alpha_2$};
            \draw [fill=ududff] (0,0) circle (1.2pt);
            \draw[color=ududff] (0.06,-0.06) node {$0$};
            \draw[color=black] (-0.35,1.4) node {$\langle\cdot,\check{\alpha_1}\rangle=0$};
            \draw[color=black] (-0.9,-0.2) node {$\langle\cdot,\check{\alpha_2}\rangle=1$};
            \draw [fill=ffqqqq] (0.5,0.28867513459481287) circle (1.2pt);
            \draw[color=ffqqqq] (0.65,0.25) node {$\omega_1$};
            \draw [fill=ffqqqq] (0,0.5773502691896257) circle (1.2pt);
            \draw[color=ffqqqq] (-0.1,0.6675848874035106) node {$\omega_2$};
            \draw[color=black] (0.85,1.4) node {$\langle\cdot,\check{\alpha_1}\rangle=1$};
            \draw[color=black] (-0.6,-0.6) node {$\langle\cdot,\check{\alpha_2}\rangle=0$};
        \end{scriptsize}
    \end{tikzpicture}
\end{center}

\vspace{0.5cm}

\begin{center}
    \begin{tikzpicture}[line cap=round,line join=round,>=triangle 45,x=2.5cm,y=2.5cm]
        \clip(-1.3,-1.3) rectangle (1.3,1.5);
        \draw [line width=0.4pt,dash pattern=on 2pt off 2pt] (0,-2) -- (0,2);
        \draw [line width=0.4pt,dash pattern=on 2pt off 2pt,domain=-2.599612871193306:3.164564097878663] plot(\x,{(-0--0.5*\x)/0.8660254037844386});
        \draw [line width=0.4pt,dash pattern=on 2pt off 2pt] (0.5,-2) -- (0.5,2);
        \draw [line width=0.4pt,dash pattern=on 2pt off 2pt,domain=-2.599612871193306:3.164564097878663] plot(\x,{(--0.28867513459481287--0.28867513459481287*\x)/0.5});
        \draw [line width=0.4pt,dash pattern=on 2pt off 2pt] (-2,1.732) -- (2,-0.5773);
        \begin{scriptsize}
            \draw [fill=black] (1,0) circle (1.2pt);
            \draw[color=black] (1.0519019508432352,0.08725485318699086) node {$\check{\alpha_1}$};
            \draw [fill=black] (-0.5,0.8660254037844386) circle (1.2pt);
            \draw[color=black] (-0.4478273510646299,0.9544896234206665) node {$\check{\alpha_2}$};
            \draw [fill=uuuuuu] (0.5,0.8660254037844386) circle (1.2pt);
            \draw[color=uuuuuu] (0.8,0.8) node {$\check{\alpha_1}+\check{\alpha_2}$};
            \draw [fill=uuuuuu] (-1,0) circle (1.2pt);
            \draw[color=uuuuuu] (-1.05,0.1) node {$-\check{\alpha_1}$};
            \draw [fill=uuuuuu] (0.5,-0.8660254037844386) circle (1.2pt);
            \draw[color=uuuuuu] (0.5498186628132109,-0.7799799170466849) node {$-\check{\alpha_2}$};
            \draw [fill=uuuuuu] (-0.5,-0.8660254037844386) circle (1.2pt);
            \draw[color=uuuuuu] (-0.5,-0.7799799170466849) node {$-\check{\alpha_1}-\check{\alpha_2}$};
            \draw [fill=ududff] (0,0) circle (1.2pt);
            \draw[color=ududff] (0.06,-0.06) node {$0$};
            \draw[color=black] (-0.35,1.4) node {$\langle{\alpha_1},\cdot\rangle=0$};
            \draw[color=black] (-0.9,-0.2) node {$\langle{\alpha_2},\cdot\rangle=1$};
            \draw [fill=ffqqqq] (0.5,0.28867513459481287) circle (1.2pt);
            \draw[color=ffqqqq] (0.68,0.285) node {$\epsilon_1^*$};
            \draw [fill=ffqqqq] (0,0.5773502691896257) circle (1.2pt);
            \draw[color=ffqqqq] (-0.1,0.73) node {$\epsilon_2^*$};
            \draw[color=black] (0.85,1.4) node {$\langle{\alpha_1},\cdot\rangle=1$};
            \draw[color=black] (-0.6,-0.6) node {$\langle{\alpha_2},\cdot\rangle=0$};
            \draw[color=black] (-1,0.87) node {$\langle{\alpha_0},\cdot\rangle=1$};
        \end{scriptsize}
    \end{tikzpicture}
\end{center}

\vspace{0.5cm}

\begin{center}
    \begin{tikzpicture}[line cap=round,line join=round,>=triangle 45,x=2.5cm,y=2.5cm]
        \clip(-1.5,-1.5) rectangle (1.5,1.6);
        \draw [line width=0.4pt,dash pattern=on 2pt off 2pt] (0,-2) -- (0,2);
        \draw [line width=0.4pt,dash pattern=on 2pt off 2pt] (-2,-2) -- (2,2);
        \draw [line width=0.4pt,dash pattern=on 2pt off 2pt] (0.5,-2) -- (0.5,2);
        \draw [line width=0.4pt,dash pattern=on 2pt off 2pt] (-2,-1) -- (2,3);
        \begin{scriptsize}
            \draw [fill=black] (1,0) circle (1.2pt);
            \draw[color=black] (1.0519019508432352,0.08725485318699086) node {$\alpha_1$};
            \draw [fill=black] (-1,1) circle (1.2pt);
            \draw[color=black] (-1,1.1) node {$\alpha_2$};
            \draw [fill=uuuuuu] (0,1) circle (1.2pt);
            \draw[color=uuuuuu] (0.1,1.1) node {$\alpha_1+\alpha_2$};
            \draw [fill=uuuuuu] (1,1) circle (1.2pt);
            \draw[color=uuuuuu] (1.05,1.1) node {$2\alpha_1+\alpha_2$};
            \draw [fill=uuuuuu] (-1,0) circle (1.2pt);
            \draw[color=uuuuuu] (-1.05,0.1) node {$-\alpha_1$};
            \draw [fill=uuuuuu] (-1,-1) circle (1.2pt);
            \draw[color=uuuuuu] (1,-0.9) node {$-\alpha_2$};
            \draw [fill=uuuuuu] (0,-1) circle (1.2pt);
            \draw[color=uuuuuu] (0,-0.9) node {$-\alpha_1-\alpha_2$};
            \draw [fill=uuuuuu] (1,-1) circle (1.2pt);
            \draw[color=uuuuuu] (-1,-0.9) node {$-2\alpha_1-\alpha_2$};
            \draw [fill=ududff] (0,0) circle (1.2pt);
            \draw[color=ududff] (0.06,-0.06) node {$0$};
            \draw[color=black] (-0.35,1.4) node {$\langle\cdot,\check{\alpha_1}\rangle=0$};
            \draw[color=black] (-1,-0.4) node {$\langle\cdot,\check{\alpha_2}\rangle=1$};
            \draw [fill=ffqqqq] (0.5,0.5) circle (1.2pt);
            \draw[color=ffqqqq] (0.65,0.5) node {$\omega_1$};
            \draw [fill=ffqqqq] (0,1) circle (1.2pt);
            \draw[color=ffqqqq] (-0.33,1.085) node {$\omega_2=$};
            \draw[color=black] (0.85,1.4) node {$\langle\cdot,\check{\alpha_1}\rangle=1$};
            \draw[color=black] (-1,-1.4) node {$\langle\cdot,\check{\alpha_2}\rangle=0$};
        \end{scriptsize}
    \end{tikzpicture}
\end{center}

\vspace{0.5cm}

\begin{center}
    \begin{tikzpicture}[line cap=round,line join=round,>=triangle 45,x=2.5cm,y=2.5cm]
        \clip(-1.5,-1.5) rectangle (1.5,1.6);
        \draw [line width=0.4pt,dash pattern=on 2pt off 2pt] (0,-2) -- (0,2);
        \draw [line width=0.4pt,dash pattern=on 2pt off 2pt] (-2,-2) -- (2,2);
        \draw [line width=0.4pt,dash pattern=on 2pt off 2pt] (0.5,-2) -- (0.5,2);
        \draw [line width=0.4pt,dash pattern=on 2pt off 2pt] (-2,-1.5) -- (2,2.5);
        \draw [line width=0.4pt,dash pattern=on 2pt off 2pt] (-2,2.5) -- (2,-1.5);
        \begin{scriptsize}
            \draw [fill=black] (1,0) circle (1.2pt);
            \draw[color=black] (1.0519019508432352,0.08725485318699086) node {$\check{\alpha_1}$};
            \draw [fill=black] (-0.5,0.5) circle (1.2pt);
            \draw[color=black] (-0.55,0.6) node {$\check{\alpha_2}$};
            \draw [fill=uuuuuu] (0,1) circle (1.2pt);
            \draw[color=uuuuuu] (0,1.1) node {$\check{\alpha_1}+2\check{\alpha_2}$};
            \draw [fill=uuuuuu] (0.5,0.5) circle (1.2pt);
            \draw[color=uuuuuu] (1.14,0.5) node {$\check{\alpha_1}+\check{\alpha_2}$};
            \draw [fill=uuuuuu] (-1,0) circle (1.2pt);
            \draw[color=uuuuuu] (-1.05,0.1) node {$-\check{\alpha_1}$};
            \draw [fill=uuuuuu] (0.5,-0.5) circle (1.2pt);
            \draw[color=uuuuuu] (0.7,-0.5) node {$-\check{\alpha_2}$};
            \draw [fill=uuuuuu] (0,-1) circle (1.2pt);
            \draw[color=uuuuuu] (0,-0.9) node {$-\check{\alpha_1}-\check{\alpha_2}$};
            \draw [fill=uuuuuu] (-0.5,-0.5) circle (1.2pt);
            \draw[color=uuuuuu] (-0.55,-0.6) node {$-2\check{\alpha_1}-\check{\alpha_2}$};
            \draw [fill=ududff] (0,0) circle (1.2pt);
            \draw[color=ududff] (0.06,-0.06) node {$0$};
            \draw[color=black] (-0.35,1.4) node {$\langle\alpha_1,\cdot\rangle=0$};
            \draw[color=black] (-1.17,-0.3) node {$\langle\alpha_2,\cdot\rangle=1$};
            \draw [fill=ffqqqq] (0.5,0.5) circle (1.2pt);
            \draw[color=ffqqqq] (0.7,0.5) node {$\epsilon_1^*=$};
            \draw [fill=ffqqqq] (0,0.5) circle (1.2pt);
            \draw[color=ffqqqq] (-0.1,0.55) node {$\epsilon_2^*$};
            \draw[color=black] (0.85,1.4) node {$\langle\alpha_1,\cdot\rangle=1$};
            \draw[color=black] (-1,-1.4) node {$\langle\alpha_2,\cdot\rangle=0$};
            \draw[color=black] (-1,1) node {$\langle\alpha_0,\cdot\rangle=1$};

        \end{scriptsize}
    \end{tikzpicture}
\end{center}